\section{操作系统与编译}
    此次实验中操作系统也是一个重要的组成部分,%
    需要根据硬件的实现情况对操作系统做出相应的修改。%

        \subsection{操作系统组成}
            此次实验中操作系统%
            bootloader部分使用贾开学长的代码%
            (与Flash访问方式有关),%
            其余均在刘亚宁学长os\_lab基础上进行修改,%
            具体修改内容之后有详细叙述。

        \subsection{时钟中断}
            在os\_lab中,启动过程中是通过MASK位进行中断屏蔽,%
            保证不会被时钟中断所打断。

            如果在硬件上没有提供对MASK位的支持,可以选择在操作系统上进行修改。%
            CP0的Compare寄存器初始化为0xFFFFFFFF,保证初始化结束之前一定不会触发时钟中断。%
            初始化完成后增加一次对clock\_intr函数的调用,%
            将Compare寄存器重填为设定的数字,之后即可触发时钟中断。

            标准MIPS时钟中断还应该在硬件上实现“读Count写入Compare”功能,%
            此问题已经在os\_lab中通过软件方式解决,不需要另作处理。

        \subsection{串口地址}
            从lab3开始,代码支持了在实验板与qemu两个平台上的运行,%
            两平台的串口地址不同。因此,在编译操作系统前需要先调用to\_thin脚本,%
            将操作系统转换为实验板上的版本。

        \subsection{异常处理向量}
            异常处理向量只有唯一的一个入口0x80000180,%
            在初始化阶段直接写入EBASE寄存器之后就不需要再进行修改了。%
            异常处理的初始化完全由操作系统完成,%
            在0x80000180存入一条jump指令,%
            跳转值alltraps函数,硬件上不需要任何的特殊处理。%

        \subsection{Flash访问}
            因为Flash数据线只有16条,%
            所以实际上Flash连续的4个byte中只有2个为有效数据,%
            因此产生了两种Flash访问方式如下:

            一种方式为每次访问Flash的4个byte,%
            得到32位数据中只有低16位有效。%
            之后在软件层进行控制,连续访问两次,%
            将结果移位拼接得到32位有效数据。%
            另一种方式为每次访问Flash在硬件上访问8个byte,%
            硬件上将两次得到的数据进行拼接,%
            对操作系统层提供与RAM访问相同的接口。

            此次实验中我们使用第一种方式。%
            因此在bootloader中使用贾开学长的代码,%
            其中包含了将访存结果移位拼接的实现。%
            除bootloader之外的操作系统使用刘亚宁学长的代码,%
            这些代码需要第二种Flash访问方式进行支持。

            lab1到lab7不需要进行任何修改,%
            因为所有Flash访问操作均是在bootloader部分完成。%
            在lab8中由于涉及到文件系统sfs.img的加载,%
            需要修改lab8/kern/fs/devs/dev\_disk0.c%
            中的disk0\_read\_blks\_nolock函数,%
            将Flash起始地址与DISK0\_BLKSIZE均乘2,%
            之后通特别外实现的memcpy\_flash将sfs.img加载到内存中。

        \subsection{编译工具}
            此次实验中采用的编译工具为%
            mips-linux-gnu工具链,%
            版本为(Sourcery CodeBench Lite 2014.05-27)%
            2.24.51.20140217。%
            编译器版本不同可能导致代码段组装顺序不同,进而可能产生问题。%
    
            可能需要修改kern\_boot函数,设置栈指针为代码段顶端的kern\_init函数。
