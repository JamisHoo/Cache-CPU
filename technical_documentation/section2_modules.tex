\section{模块设计}
    \subsection{取指令模块}
        \subsubsection{端口说明}
                    \begin{tabularx}{\textwidth}{lll}
            \toprule
            端口名          & 端口方向  & 端口类型 \\
            \cmidrule(l){2-3}
            &
            \multicolumn{2}{X}{端口描述} \\
            \midrule
            clk             & in        & std\_logic \\
            \cmidrule(l){2-3}
            &
            \multicolumn{2}{X}{
                CPU时钟信号
            } \\
            \midrule
            rst             & in        & std\_logic \\
            \cmidrule(l){2-3}
            &
            \multicolumn{2}{X}{
                初始化信号,在CPU启动时使用。
            } \\
            \midrule
            state           & in        & status(自定义状态集合) \\
            \cmidrule(l){2-3}
            &
            \multicolumn{2}{X}{
                CPU当前状态
            } \\
            \midrule
            PCSrc           & in        & std\_logic\_vector(31 downto 0) \\
            \cmidrule(l){2-3}
            &
            \multicolumn{2}{X}{
                说明:非异常状态下的指令地址。

                来源:WB模块。

                到达时间:当前指令InsF上升沿之前。

                产生时间:上一条指令WB上升沿之后。
            } \\
            \midrule
            EBase           & in        & std\_logic\_vector(31 downto 0) \\
            \cmidrule(l){2-3}
            &
            \multicolumn{2}{X}{
                说明:异常处理基地址。

                来源:CP0模块。

                到达时间:当前指令InsF上升沿到来之前。
            } \\
            \midrule
            EPC             & in        & std\_logic\_vector(31 downto 0) \\
            \cmidrule(l){2-3}
            &
            \multicolumn{2}{X}{
                说明:ERET指令的返回地址。

                来源:CP0模块。

                到达时间:当前指令InsF上升沿到来之前。
            } \\
            \midrule
            pc\_sel         & in        & std\_logic\_vector(1 downto 0) \\
            \cmidrule(l){2-3}
            &
            \multicolumn{2}{X}{
                pc\_sel(1):

                    说明:eret\_enable,使能信号。

                    来源:ID模块。

                    到达时间:当前指令InsF上升沿到来之前。

                    产生时间:上一条指令ID上升沿之后。
            } \\
            &
            \multicolumn{2}{X}{
                pc\_sel(0):

                    说明:pc\_control,判断是否为异常状态。

                    来源:异常模块。

                    到达时间:当前指令InsF上升沿到来之前。
            } \\
            \midrule
            PC              & out       & std\_logic\_vector(31 downto 0) \\
            \cmidrule(l){2-3}
            &
            \multicolumn{2}{X}{
                说明:PC寄存器,时序逻辑。

                产生时间:当前指令InsF上升沿到来之后。

                有效时间:直到下一条指令的IF阶段。
            } \\
            \midrule
            PCmmu           & out       & std\_logic\_vector(31 downto 0) \\
            \cmidrule(l){2-3}
            &
            \multicolumn{2}{X}{
                说明:为MMU单元提供的PC,组合逻辑。

                产生时间:当前指令InsF上升沿到来之前。

                有效时间:直到当前指令IF阶段结束。
            } \\
            \bottomrule
        \end{tabularx}
 
        \subsubsection{内部实现}
            需要的数据有PcSrc、EBase、EPC。

            PcSrc产生于上一条指令的WB阶段。%
            EBase为固定值,直接从CP0部分连接过来。%
            EPC在异常阶段写入,直接从CP0部分连接过来。%
            pc\_sel为上一条指令的解码阶段产生。%
            因此,所有的数据都能够在InsF时钟上升沿之前准备完毕。

            内部分为两个部分对PC进行计算:%
            首先为组合逻辑部分,需要在InsF时钟上升沿到来之前为MMU计算出PC值,%
            因此建立一个process,敏感信号为全部的PC选择信息,%
            即时计算出PC值,连接到MMU部分,%
            使得MMU能够在InsF上升沿进行取指令的操作。%
            其次为时序逻辑部分,在InsF时钟上升沿时,%
            对PC进行选择,选择方式与PCmmu相同。%
            此process产生的PC,在当前指令的全部周期有效,%
            是计算RPC、branch、jump的地址的基础。%
    \subsection{指令解析模块}
        \subsubsection{端口说明}
                    \begin{tabularx}{\textwidth}{lll}
            \toprule
            端口名          & 端口方向  & 端口类型 \\
            \cmidrule(l){2-3}
            &
            \multicolumn{2}{X}{端口描述} \\
            \midrule
            clk             & in        & std\_logic \\
            \cmidrule(l){2-3}
            &
            \multicolumn{2}{X}{
                原作者很懒,没有写端口描述。
            } \\
            \midrule
            state           & in        & std\_logic\_vector(3 downto 0) \\
            \cmidrule(l){2-3}
            &
            \multicolumn{2}{X}{
                原作者很懒,没有写端口描述。
            } \\
            \midrule
            instruction     & in        & std\_logic\_vector(31 downto 0) \\
            \cmidrule(l){2-3}
            &
            \multicolumn{2}{X}{
                说明:当前指令。

                来源:MMU模块。

                到达时间:当前指令InsD上升沿之前。

                产生时间:当前指令InsF上升沿之后。
            } \\
            \midrule
            instr\_out      & out       & std\_logic\_vector(31 downto 0) \\
            \cmidrule(l){2-3}
            &
            \multicolumn{2}{X}{
                说明:指令寄存器,除三个寄存器的编号,%
                其他所有控制线均从此产生。%
                之后周期中如果需要用到指令也从此处获得。

                产生时间:当前指令InsD上升沿之后。
            } \\
            \midrule
            rs\_addr        & out       & std\_logic\_vector(4 downto 0) \\
            \cmidrule(l){2-3}
            &
            \multicolumn{2}{X}{
                说明:通用寄存器编号1,在InsD阶段需要读取到值。

                产生时间:当前指令InsD上升沿之前。
            } \\
            \midrule
            rt\_addr        & out       & std\_logic\_vector(4 downto 0) \\
            \cmidrule(l){2-3}
            &
            \multicolumn{2}{X}{
                说明:通用寄存器编号2、写入寄存器编号,在InsD阶段需要读取到值。

                产生时间:当前指令InsD上升沿之前。
            } \\
            \midrule
            rd\_addr        & out       & std\_logic\_vector(4 downto 0) \\
            \cmidrule(l){2-3}
            &
            \multicolumn{2}{X}{
                说明:CP0寄存器编号、写入寄存器编号,在InsD阶段需要读取到值。

                产生时间:当前指令InsD上升沿之前。
            } \\
            \midrule
            pc\_op          & out       & std\_logic\_vector(1 downto 0) \\
            \cmidrule(l){2-3}
            &
            \multicolumn{2}{X}{
                说明:PCSrc选择器,正常状态下PC的选择方式。%
                输出到WB模块,4选1数据选择器的控制信号,选择正确的PC。

                产生时间:当前指令InsD上升沿之后。

                有效时间:下一条指令InsD上升沿之后。 
            } \\
            \midrule
            eret\_enable    & out       & std\_logic \\

            \cmidrule(l){2-3}
            &
            \multicolumn{2}{X}{
                说明:ERET使能,对PC进行选择。%
                专门对ERET指令使用,输出到IFetch模块。

                产生时间:当前指令InsD上升沿之后。

                有效时间:下一条指令InsD上升沿之后。 
            } \\
            \midrule
            comp\_op        & out       & std\_logic\_vector(2 downto 0) \\
            \cmidrule(l){2-3}
            &
            \multicolumn{2}{X}{
                说明:比较信号,branch指令的跳转条件。%
                输出到WB模块,如果为branch系列指令则通过此信号进行选择。

                产生时间:当前指令InsD上升沿之后。

                有效时间:下一条指令InsD上升沿之后。 
            } \\
            \midrule
            imme            & out       & std\_logic\_vector(31 downto 0) \\
            \cmidrule(l){2-3}
            &
            \multicolumn{2}{X}{
                说明:32位立即数,针对不同指令的需求产生。%
                立即数本身作为ALUSrc的来源之一。

                产生时间:当前指令InsD上升沿之后。

                有效时间:下一条指令InsD上升沿之后。 
            } \\
            \midrule
            alu\_ops        & out       & std\_logic\_vector(8 downto 0) \\
            \cmidrule(l){2-3}
            &
            \multicolumn{2}{X}{
                说明:控制ALU模块。

                ALUSrcA(8 downto 7):ALU第一输入的选择信号,%
                四选一数据选择。

                ALUSrcB(6 downto 5):ALU第二输入的选择信号,%
                四选一数据选择。

                ALUOp(4 downto 0):ALU操作,%
                暂定4位,可以扩展为5位,详细说明见表格。

                产生时间:当前指令InsD上升沿之后。

                有效时间:下一条指令InsD上升沿之后。 
            } \\
            \midrule
            mem\_op         & out       & std\_logic\_vector(2 downto 0) \\
            \cmidrule(l){2-3}
            &
            \multicolumn{2}{X}{
                说明:控制MEM模块。

                MEMRead(2):是否可读内存。

                MEMWrite(1):是否可写内存。

                MEMValue(0):选择写入内存的值,%
                寄存器的数据或者SB指令处理之后的数据。

                产生时间:当前指令InsD上升沿之后。

                有效时间:下一条指令InsD上升沿之后。
            } \\
            \midrule
            wb\_op          & out       & std\_logic\_vector(5 downto 0) \\
            \cmidrule(l){2-3}
            &
            \multicolumn{2}{X}{
                说明:控制WB模块。

                RegDst(5 downto 4):写回寄存器编号,%
                            rt或者rd或者31号寄存器。
                RegValue(3 downto 1):写回寄存器的内容,%
                                信号详细说明见表格。
                RegWrite(0):寄存器是否可写。

                产生时间:当前指令InsD上升沿之后。

                有效时间:下一条指令InsD上升沿之后。 
            } \\
            \midrule
            cp0\_op         & out       & std\_logic\_vector(1 downto 0) \\
            \cmidrule(l){2-3}
            &
            \multicolumn{2}{X}{
                说明:控制CP0模块。

                EPCValue(1):异常产生时,EPC写入的内容,%
                选择写入PC或者PC+4。

                CP0Write(0):CP0寄存器是否可写。

                产生时间:当前指令InsD上升沿之后。

                有效时间:下一条指令InsD上升沿之后 
            } \\
            \midrule
            tlbwi\_enable   & out       & std\_logic \\
            \cmidrule(l){2-3}
            &
            \multicolumn{2}{X}{
                说明:TLB写使能。

                产生时间:当前指令InsD上升沿之后。

                有效时间:下一条指令InsD上升沿之后。
            } \\
            \bottomrule
            \end{tabularx}

        \subsubsection{内部实现}
            需要的数据有instruction,%
            产生于当前指令InsF上升沿之后,能够在当前指令InsD上升沿之前到达。

            内部实现分为两部分:%
            \begin{enumerate}
            \item
            在InsD阶段除了解码指令之外,%
            还需要读取通用寄存器和CP0寄存器的值。%
            需要在InsD上升沿到来之前,将三个寄存器编号发送给寄存器堆。%
            因此将instruction的三个5位的寄存器编号直接连接到输出的rs、rt、rd部分。%
            \item
            其他控制线的生成均通过时钟驱动,%
            在IDEcode有相应寄存器,连接至输出端。%
            在InsD时钟上升沿之后,根据指令解码产生控制信号。%
            产生的控制信号根据需要,%
            输出到ALU、WB、MEM、CP0、IFetch、MMU模块。
            \end{enumerate}

    \subsection{ALU模块}
        \subsubsection{端口说明}
                    \begin{tabularx}{\textwidth}{lll}
            \toprule
            端口名          & 端口方向 & 端口类型 \\
            \cmidrule(l){2-3}
                            & \multicolumn{2}{l}{端口描述} \\
            \midrule
            clk             & in       & std\_logic \\
            \cmidrule(l){2-3}
            &
            \multicolumn{2}{X}{CPU时钟信号。} \\
            \midrule
            enable          & in       & std\_logic \\
            \cmidrule(l){2-3}
            & 
            \multicolumn{2}{X}{
            ALU模块是否使能,为1时工作,为0时不工作,%
            需要在clk时钟上升沿之前准备好,%
            并保持到clk时钟上升沿之后一极短时间。} \\
            \midrule
            rs\_value        & in       & std\_logic\_vector(31 downto 0) \\
            \cmidrule(l){2-3}
            &
            \multicolumn{2}{X}{
            来自通用寄存器堆的第一个值。需要在clk时钟上升沿之前准备好,%
            并保持到clk时钟上升沿之后一极短时间。} \\
            \midrule
            rt\_value        & in       & std\_logic\_vector(31 downto 0) \\
            \cmidrule(l){2-3}
            & 
            \multicolumn{2}{X}{
            来自通用寄存器堆的第二个值。需要在clk时钟上升沿之前准备好,%
            并保持到clk时钟上升沿之后一极短时间。} \\
            \midrule
            imme            & in       & std\_logic\_vector(31 downto 0) \\
            \cmidrule(l){2-3}
            &
            \multicolumn{2}{X}{
            指令中包含的立即数,来自指令解析模块。%
            需要在clk时钟上升沿之前准备好,%
            并保持到clk时钟上升沿之后一极短时间。} \\
            \midrule
            cp0\_value      & in       & std\_logic\_vector(31 downto 0) \\
            \cmidrule(l){2-3}
            &
            \multicolumn{2}{X}{
            来自cp0寄存器,mfc0指令需要。%
            需要在clk时钟上升沿之前准备好,%
            并保持到clk时钟上升沿之后一极短时间。} \\
            \midrule
            state           & in       & std\_logic\_vector(3 downto 0) \\
            \cmidrule(l){2-3}
            &
            \multicolumn{2}{X}{
            来自状态控制模块,用来指示当前处于工作状态的模块。%
            若当前非ALU工作状态,%
            则任何外部输入都不会对ALU的hi、lo寄存器以及ALU的输出造成修改。%
            需要在clk时钟上升沿之前准备好,%
            并保持到clk时钟上升沿之后一极短时间。} \\
            \midrule
            alu\_op         & in       & std\_logic\_vector(4 downto 0) \\
            \cmidrule(l){2-3}
            &
            \multicolumn{2}{X}{
            来自指令解析模块的ALU运算操作符。%
            需要在clk时钟上升沿之前准备好,%
            并保持到clk时钟上升沿之后一极短时间。%
            各操作符代表的意义为%
            (A、B分别代表经过alu\_srcA、alu\_srcB选择后的值,%
            result代表alu\_result,lo、hi代表乘法寄存器):%
            } \\
            &
            \multicolumn{2}{X}{
            00000 result = A + B

            00001 result = A - B

            00010 result = A - B(比较大小,实际做减法)

            00011 result = A \& B
            } \\
            &
            \multicolumn{2}{X}{
            00100 result = A | B

            00101 result = A \textasciicircum \ B

            00110 result = ~(A | B)

            00111 result = B << A
            } \\
            &
            \multicolumn{2}{X}{
            01000 result = B >> A(算术右移)

            01001 result = B >> A(逻辑右移)
            } \\
            &
            \multicolumn{2}{X}{
            01010 result = A < B?(有符号比较,结果真时最低位输出1,否则输出0,其他位总是输出0)

            01011 result = A < B?(无符号比较,结果真时最低位输出1,否则输出0,其他位总是输出0)
            } \\
            &
            \multicolumn{2}{X}{
            10000 hi\_lo = A * B(补码乘法)

            10001 result = lo

            10010 result = hi
            } \\
            &
            \multicolumn{2}{X}{
            10011 lo = A

            10100 hi = A} \\
            \midrule
            alu\_srcA & in & std\_logic\_vector(1 downto 0) \\
            \cmidrule(l){2-3}
            &
            \multicolumn{2}{X}{
            ALU的第一个操作数的选择码,%
            当值为“00”时选取rs\_value,%
            当值为“01”时选取imme,当值为“10”时选取cp0\_value,%
            当值为“11”时选取立即数16。需要在clk时钟上升沿之前准备好,%
            并保持到clk时钟上升沿之后一极短时间。
            } \\
            \midrule
            alu\_srcB & in & std\_logic\_vector(1 downto 0) \\
            \cmidrule(l){2-3}
            &
            \multicolumn{2}{X}{
            ALU的第二个操作数的选择码,当值为“00”时选取rt\_value,%
            当值为“01”时选取imme,当值为“10”时选取cp0\_value。%
            需要在clk时钟上升沿之前准备好,%
            并保持到clk时钟上升沿之后一极短时间。
            } \\
            \midrule
            alu\_result & out & std\_logic\_vector(31 downto 0) \\
            \cmidrule(l){2-3}
            &
            \multicolumn{2}{X}{
            ALU的输出,在下一个时钟上升沿之前准备好,%
            并保持直到下一次能使ALU输出改变%
            (state为ALU工作状态、enable为1、alu\_op非写hi、lo寄存器)%
            的时钟上升沿。
            } \\
            \bottomrule
            \end{tabularx}

        \subsubsection{内部实现}
            每次时钟上升沿到来时,检查enable、state,%
            若不是ALU工作状态则不进行任何操作。%
            根据alu\_srcA选择第一个操作数,根据alu\_srcB选择第二个操作数,%
            根据操作码进行相应的运算,将结果输出或保存到hi、lo寄存器中。

            注意乘法运算需要较多的时间,因此若在某一个时钟上升沿进行乘法运算,%
            不能认为在下一个时钟上升沿就能在hi、lo寄存器中取到正确的结果。%
            一般来说,在12.5MHz时钟频率下进行乘法运算需要大约2个时钟周期,%
            那么如果在多周期CPU上,%
            可以保证两条连续的乘法、取hi(lo)寄存器指令能得到正确的结果,%
            但是在流水线CPU上不能保证。%
            乘法运算时间需要根据不同的硬件平台、时钟频率进行测量,%
            不能一概而论。
