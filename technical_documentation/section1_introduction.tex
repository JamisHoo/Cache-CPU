\section{引言}
    \subsection{编写目的}
        在此前编写的需求文档中,已经明确了此次联合实验预期达到的目标,实验中需要完成的各部分工作,
        也对实验中需要用到的关键技术做了简要的原理性说明,此次实验的前期准备工作的需求文档中基本体现。

        进入实际的代码开发阶段,VHDL代码的编写需要更加详细的接口,更加精准的功能说明,更加细化的流程控制。
        从前的需求文档已经不足以对开发过程进行具体的指导了,需要一份更加详细的设计文档。

        因此,为了指导代码的实际开发过程,编写此设计文档。

        设计文档预期读者为任务提出者:刘卫东老师、李山山老师、白晓颖老师。
        未来需要完成此实验的同学也可参考本文档进行设计。

    \subsection{背景}
        系统名称:32位MIPS处理器

        任务提出者:
        \begin{minipage}[t]{0.5\linewidth}
        计原课:刘卫东老师、李山山老师

        软工课:白晓颖老师
        \end{minipage}
        

        开发者:
        \begin{minipage}[t]{0.5\linewidth}
        计23 李天润

        计23 胡津铭

        计23 孙皓
        \end{minipage}

    \subsection{参考资料}
        实验指导文档

        OsLab实验参考文档

        计算机组成原理综合实验报告 贾开

        《计算机组成和设计 硬件/软件接口》

        《See MIPS Run》


