        \begin{tabularx}{\textwidth}{lll}
            \toprule
            端口名          & 端口方向  & 端口类型 \\
            \cmidrule(l){2-3}
            &
            \multicolumn{2}{X}{端口描述} \\
            \midrule
            clk             & in        & std\_logic \\
            \cmidrule(l){2-3}
            &
            \multicolumn{2}{X}{
                CPU时钟信号
            } \\
            \midrule
            rst             & in        & std\_logic \\
            \cmidrule(l){2-3}
            &
            \multicolumn{2}{X}{
                初始化信号,在CPU启动时使用。
            } \\
            \midrule
            state           & in        & std\_logic\_vector(3 downto 0) \\
            \cmidrule(l){2-3}
            &
            \multicolumn{2}{X}{
                说明:CPU状态机信号
            } \\
            \midrule
            if\_addr        & in        & std\_logic\_vector(31 downto 0) \\
            \cmidrule(l){2-3}
            &
            \multicolumn{2}{X}{
                说明:取指令地址

                来源:IFetch模块

                到达时间:取指令时钟上升沿之前
            } \\
            \midrule
            instruction        & out        & std\_logic\_vector(31 downto 0) \\
            \cmidrule(l){2-3}
            &
            \multicolumn{2}{X}{
                说明:取指令阶段得到的32位指令

                产生时间:访存阶段结束之后,根据ready位进行判断
            } \\
            \midrule
            virtual\_addr & in      & std\_logic\_vector(31 downto 0) \\
            \cmidrule(l){2-3}
            &
            \multicolumn{2}{X}{
                说明:访存阶段的虚拟地址

                来源:MEM模块

                到达时间:访存阶段时钟上升沿之前
            } \\
            \midrule
            data\_in & in     & std\_logic\_vector(31 downto 0) \\
            \cmidrule(l){2-3}
            &
            \multicolumn{2}{X}{
                说明:访存阶段的写入数据

                来源:MEM模块

                到达时间:访存阶段时钟上升沿之前
            } \\
            \midrule
            read\_enable       & in        & std\_logic \\
            \cmidrule(l){2-3}
            &
            \multicolumn{2}{X}{
                说明:内存读使能,需要进一步处理

                来源:MEM模块

                到达时间:访存阶段时钟上升沿之前
            } \\
            \midrule
            write\_enable & in      & std\_logic \\
            \cmidrule(l){2-3}
            &
            \multicolumn{2}{X}{
                说明:内存写使能,需要进一步处理

                来源:MEM模块

                到达时间:访存阶段时钟上升沿之前
            } \\
            \midrule
            data\_out      & out        & std\_logic\_vector(31 downto 0) \\
            \cmidrule(l){2-3}
            &
            \multicolumn{2}{X}{
                说明:访存阶段输出的结果

                产生时间:访存阶段结束之前,在ready位置1的时候
            } \\
            \midrule
            ready   & out        & std\_logic \\
            \cmidrule(l){2-3}
            &
            \multicolumn{2}{X}{
                说明:标志位,访存是否结束

                产生时间:访存阶段结束之前
            } \\
            \midrule
            serial\_int      & out       & std\_logic \\
            \cmidrule(l){2-3}
            &
            \multicolumn{2}{X}{
                说明:串口中断信号,输出到异常模块

                产生时间:外部中断,任意时间均可产生
            } \\
            \midrule
            exc\_code      & out       & std\_logic\_vector(2 downto 0) \\
            \cmidrule(l){2-3}
            &
            \multicolumn{2}{X}{
                说明:异常信号,输出到异常模块

                产生时间:访存阶段第一个下降沿,下一个下降沿就被清空

                000 无异常产生
            } \\
            &
            \multicolumn{2}{X}{
                001 TLB修改异常

                010 TLB缺失(读)

                011 TLB缺失(写)

                100 地址不对齐(读)

                101 地址不对齐(写)
            } \\
            \midrule
            tlb\_write\_struct      & in       & std\_logic\_vector(66 downto 0) \\
            \cmidrule(l){2-3}
            &
            \multicolumn{2}{X}{
                说明:TLB写结构,包含一个TLB表项所需的所有信息

                来源:CP0模块

                到达时间:始终从CP0模块连接到MMU,始终处于可读状态
            } \\
            \midrule
            tlb\_write\_enable      & in       & std\_logic \\
            \cmidrule(l){2-3}
            &
            \multicolumn{2}{X}{
                说明:TLB写使能,判断是否写入

                来源:指令解码模块
                
                到达时间:指令解码阶段上升沿之后
            } \\
            \midrule
            align\_type      & in       & std\_logic\_vector(1 downto 0) \\
            \cmidrule(l){2-3}
            &
            \multicolumn{2}{X}{
                说明:地址对齐方式,配合地址不对齐异常使用

                来源:指令解码模块
                
                到达时间:指令解码阶段上升沿之后
            } \\
            \midrule
            to\_physical\_addr      & out       & std\_logic\_vector(23 downto 0) \\
            \cmidrule(l){2-3}
            &
            \multicolumn{2}{X}{
                说明:给物理访存模块的地址,包括了访存类型以及地址

                产生时间:访存阶段第一个下降沿之前
            } \\
            \midrule
            to\_physical\_data      & out       & std\_logic\_vector(31 downto 0) \\
            \cmidrule(l){2-3}
            &
            \multicolumn{2}{X}{
                说明:给物理访存模块的数据,写内存或串口时需要

                产生时间:访存阶段第一个下降沿之前
            } \\
            \midrule
            to\_physical\_read\_enable      & out       & std\_logic \\
            \cmidrule(l){2-3}
            &
            \multicolumn{2}{X}{
                说明:给物理访存模块的读使能,异常状态下直接置0

                产生时间:访存阶段第一个下降沿之前
            } \\
            \midrule
            to\_physical\_write\_enable      & out       & std\_logic \\
            \cmidrule(l){2-3}
            &
            \multicolumn{2}{X}{
                说明:给物理访存模块的写使能,异常状态下直接置0

                产生时间:访存阶段第一个下降沿之前
            } \\
            \midrule
            from\_physical\_data      & in       & std\_logic\_vector(31 downto 0) \\
            \cmidrule(l){2-3}
            &
            \multicolumn{2}{X}{
                说明:从物理访存模块返回的数据,经过处理直接输出

                产生模块:物理访存模块

                到达时间:访存时间长度不定,在ready位置1之后
            } \\
            \midrule
            from\_physical\_ready     & in       & std\_logic \\
            \cmidrule(l){2-3}
            &
            \multicolumn{2}{X}{
                说明:从物理访存模块返回的状态位,说明访存是否完成

                产生模块:物理访存模块

                产生时间:物理访存结束之后
            } \\
            \midrule
            from\_physical\_serial      & in       & std\_logic \\
            \cmidrule(l){2-3}
            &
            \multicolumn{2}{X}{
                说明:串口状态,如果串口有数据则为1

                产生模块:物理访存模块

                产生时间:串口为外部中断,因此可能在任意时间产生
            } \\
            \bottomrule
        \end{tabularx}
