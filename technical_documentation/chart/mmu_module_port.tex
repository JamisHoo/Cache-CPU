        \begin{tabularx}{\textwidth}{lll}
            \toprule
            端口名          & 端口方向  & 端口类型 \\
            \cmidrule(l){2-3}
            &
            \multicolumn{2}{X}{端口描述} \\
            \midrule
            clk             & in        & std\_logic \\
            \cmidrule(l){2-3}
            &
            \multicolumn{2}{X}{
                CPU时钟信号
            } \\
            \midrule
            state           & in        & std\_logic\_vector(3 downto 0) \\
            \cmidrule(l){2-3}
            &
            \multicolumn{2}{X}{
                说明:CPU状态机信号
            } \\
            \midrule
            mmu\_clock      & in        & std\_logic \\
            \cmidrule(l){2-3}
            &
            \multicolumn{2}{X}{
                说明:访存相关时钟信号。

                内存的时钟频率和CPU工作的时钟频率并不相同,%
                内存的时钟频率比CPU快。
            } \\
            \midrule
            mem\_read\_enable & in      & std\_logic \\
            \cmidrule(l){2-3}
            &
            \multicolumn{2}{X}{
                说明:内存读使能

                来源:MEM模块

                到达时间:访存阶段时钟上升沿之前
            } \\
            \midrule
            mem\_write\_enable & in     & std\_logic \\
            \cmidrule(l){2-3}
            &
            \multicolumn{2}{X}{
                说明:内存写使能

                来源:MEM模块

                到达时间:访存阶段时钟上升沿之前
            } \\
            \midrule
            if\_addr        & in        & std\_logic\_vector(31 downto 0) \\
            \cmidrule(l){2-3}
            &
            \multicolumn{2}{X}{
                说明:取指令地址

                来源:IFetch模块

                到达时间:取指令时钟上升沿之前
            } \\
            \midrule
            mem\_addr       & in        & std\_logic\_vector(31 downto 0) \\
            \cmidrule(l){2-3}
            &
            \multicolumn{2}{X}{
                说明:访存阶段(不包括取指令)的访存地址

                来源:MEM模块

                到达时间:访存阶段时钟上升沿之前
            } \\
            \midrule
            mem\_write\_value & in      & std\_logic\_vector(31 downto 0) \\
            \cmidrule(l){2-3}
            &
            \multicolumn{2}{X}{
                说明:内存写入值

                来源:MEM模块

                到达时间:访存阶段时钟上升沿之前
            } \\
            \midrule
            cp0\_value      & in        & std\_logic\_vector(159 downto 0) \\
            \cmidrule(l){2-3}
            &
            \multicolumn{2}{X}{
                说明:来自CP0的数据,支持TLBWI指令

                来源:CP0模块

                到达时间:TLBWI指令执行阶段时钟上升沿之前
            } \\
            \midrule
            tlbwi\_enable   & in        & std\_logic \\
            \cmidrule(l){2-3}
            &
            \multicolumn{2}{X}{
                说明:TLB写使能

                来源:IDecode模块

                到达时间:当前指令解码时钟上升沿之后
            } \\
            \midrule
            mem\_value      & out       & std\_logic\_vector(31 downto 0) \\
            \cmidrule(l){2-3}
            &
            \multicolumn{2}{X}{
                说明:读取内存的结果

                产生时间:访存阶段时钟下降沿之后
            } \\

            \bottomrule
        \end{tabularx}
