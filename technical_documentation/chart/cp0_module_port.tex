        \begin{tabularx}{\textwidth}{lll}
            \toprule
            端口名          & 端口方向  & 端口类型 \\
            \cmidrule(l){2-3}
            &
            \multicolumn{2}{X}{端口描述} \\
            \midrule
            clk             & in        & std\_logic \\
            \cmidrule(l){2-3}
            &
            \multicolumn{2}{X}{
                CPU时钟信号
            } \\
            \midrule
            state           & in        & status(自定义状态集合) \\
            \cmidrule(l){2-3}
            &
            \multicolumn{2}{X}{
                CPU当前状态
            } \\
            \midrule
	        normal\_cp0\_in & in        & std\_logic\_vector(37 downto 0) \\
            \cmidrule(l){2-3}
            &
            \multicolumn{2}{X}{
	            指令引发的CP0读写操作的输入,格式上,%
                37位为写使能,36-32位为地址,31-0位为数据。

	            读写均在时钟上升沿触发,%
                因此要求数据在时钟上升沿之前准备好。

	            状态方面,读操作发生在ID阶段的上升沿,%
                写操作发生在ALU阶段的上升沿。
            } \\
            \midrule
	        bad\_v\_addr\_in & in       & std\_logic\_vector(31 downto 0) \\
            \cmidrule(l){2-3}
            &
            \multicolumn{2}{X}{
	            异常发生时写入bad\_v\_addr\_in的数据,%
                要求数据在时钟上升沿之前准备好。%
                使能为interrupt\_start\_in。
            } \\
            \midrule
	        entry\_hi\_in   & in        & std\_logic\_vector(19 downto 0) \\
            \cmidrule(l){2-3}
            &
            \multicolumn{2}{X}{
	            异常发生时写入entry\_hi\_in高20位的数据,%
                要求数据在时钟上升沿之前准备好。%
                使能为interrupt\_start\_in。
            } \\
            \midrule
	        interrupt\_start\_in & in   & std\_logic \\
            \cmidrule(l){2-3}
            &
            \multicolumn{2}{X}{
	            异常写入的使能,控制异常数据的写入,%
                并将status(1)制1.
            } \\
            \midrule
	        cause\_in       & in        & std\_logic\_vector(4 downto 0) \\
            \cmidrule(l){2-3}
            &
            \multicolumn{2}{X}{
	            异常发生时写入cause的6-2位的数据,%
                要求数据在时钟上升沿之前准备好。%
                使能为interrupt\_start\_in。
            } \\
            \midrule
	        epc\_in         & in        & std\_logic\_vector(31 downto 0) \\
            \cmidrule(l){2-3}
            &
            \multicolumn{2}{X}{
	            异常发生时写入epc的数据,%
                要求数据在时钟上升沿之前准备好。%
                使能为interrupt\_start\_in。
            } \\
            \midrule
	        eret\_enable    & in        & std\_logic \\
            \cmidrule(l){2-3}
            &
            \multicolumn{2}{X}{
	            eret的使能信号,%
                将status(1)置0.优先于interrupt\_start\_in起效。

	            请注意,这一数据应当在ALU阶段上升沿之前准备好。
            } \\
            \midrule
	        cp0\_e          & in        & std\_logic \\
	        \cmidrule(l){2-3}
            &
            \multicolumn{2}{X}{
                CP0部分的使能信号。
            } \\
            \midrule
	        addr\_value     & out       & std\_logic\_vector(31 downto 0) \\
            \cmidrule(l){2-3}
            &
            \multicolumn{2}{X}{
	            normal\_cp0\_in读操作时读出的数据,%
                ID阶段上升沿之后起效,%
                下一ID阶段上升沿之前均不变。

	            初始值为全0.
            } \\
            \midrule
	        all\_regs       & out       & std\_logic\_vector(1023 downto 0) \\
            \cmidrule(l){2-3}
            &
            \multicolumn{2}{X}{
	            即时输出全部CP0寄存器的值,%
                CP0寄存器数值被修改的时间内不保证数值稳定。
            } \\
            \midrule
	        compare\_interrupt & out    & std\_logic \\
            \cmidrule(l){2-3}
            &
            \multicolumn{2}{X}{
	            clock寄存器与compare寄存器数值相同之后被置1;%
                修改compare寄存器的值后置0%
                且该周期不比较clock寄存器与compare寄存器的值。%
	            缺少将此值恢复为0的信号。%
	            变为1后,若不手动恢复为0或者修改compare寄存器,%
                则1保持。此输出值被检测到为1时触发中断,%
                任一上升沿均可检测,不需太早处理。
            } \\

            \bottomrule
        \end{tabularx}
