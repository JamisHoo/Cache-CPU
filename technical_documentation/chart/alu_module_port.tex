        \begin{tabularx}{\textwidth}{lll}
            \toprule
            端口名          & 端口方向 & 端口类型 \\
            \cmidrule(l){2-3}
                            & \multicolumn{2}{l}{端口描述} \\
            \midrule
            clk             & in       & std\_logic \\
            \cmidrule(l){2-3}
            &
            \multicolumn{2}{X}{CPU时钟信号。} \\
            \midrule
            rs\_value        & in       & std\_logic\_vector(31 downto 0) \\
            \cmidrule(l){2-3}
            &
            \multicolumn{2}{X}{
            来自通用寄存器堆的第一个值。需要在clk时钟上升沿之前准备好,%
            并保持到clk时钟上升沿之后一极短时间。} \\
            \midrule
            rt\_value        & in       & std\_logic\_vector(31 downto 0) \\
            \cmidrule(l){2-3}
            & 
            \multicolumn{2}{X}{
            来自通用寄存器堆的第二个值。需要在clk时钟上升沿之前准备好,%
            并保持到clk时钟上升沿之后一极短时间。} \\
            \midrule
            imme            & in       & std\_logic\_vector(31 downto 0) \\
            \cmidrule(l){2-3}
            &
            \multicolumn{2}{X}{
            指令中包含的立即数,来自指令解析模块。%
            需要在clk时钟上升沿之前准备好,%
            并保持到clk时钟上升沿之后一极短时间。} \\
            \midrule
            cp0\_value      & in       & std\_logic\_vector(31 downto 0) \\
            \cmidrule(l){2-3}
            &
            \multicolumn{2}{X}{
            来自cp0寄存器,mfc0指令需要。%
            需要在clk时钟上升沿之前准备好,%
            并保持到clk时钟上升沿之后一极短时间。} \\
            \midrule
            state           & in       & status(自定义状态集合) \\
            \cmidrule(l){2-3}
            &
            \multicolumn{2}{X}{
            来自状态控制模块,用来指示当前处于工作状态的模块。%
            若当前非ALU工作状态,%
            则任何外部输入都不会对ALU的hi、lo寄存器以及ALU的输出造成修改。%
            需要在clk时钟上升沿之前准备好,%
            并保持到clk时钟上升沿之后一极短时间。} \\
            \midrule
            alu\_op         & in       & std\_logic\_vector(4 downto 0) \\
            \cmidrule(l){2-3}
            &
            \multicolumn{2}{X}{
            来自指令解析模块的ALU运算操作符。%
            需要在clk时钟上升沿之前准备好,%
            并保持到clk时钟上升沿之后一极短时间。%
            各操作符代表的意义为%
            (A、B分别代表经过alu\_srcA、alu\_srcB选择后的值,%
            result代表alu\_result,lo、hi代表乘法寄存器):%
            } \\
            &
            \multicolumn{2}{X}{
            00000 result = A + B

            00001 result = A - B

            00010 result = A - B(比较大小,实际做减法)

            00011 result = A \& B
            } \\
            &
            \multicolumn{2}{X}{
            00100 result = A | B

            00101 result = A \textasciicircum \ B

            00110 result = $\sim$(A | B)

            00111 result = B << A
            } \\
            &
            \multicolumn{2}{X}{
            01000 result = B >> A(算术右移)

            01001 result = B >> A(逻辑右移)
            } \\
            &
            \multicolumn{2}{X}{
            01010 result = A < B?(有符号比较,结果真时最低位输出1,否则输出0,其他位总是输出0)

            01011 result = A < B?(无符号比较,结果真时最低位输出1,否则输出0,其他位总是输出0)
            } \\
            &
            \multicolumn{2}{X}{
            10000 hi\_lo = A * B(补码乘法)

            10001 result = lo

            10010 result = hi
            } \\
            &
            \multicolumn{2}{X}{
            10011 lo = A

            10100 hi = A} \\
            \midrule
            alu\_srcA & in & std\_logic\_vector(1 downto 0) \\
            \cmidrule(l){2-3}
            &
            \multicolumn{2}{X}{
            ALU的第一个操作数的选择码,%
            当值为“00”时选取rs\_value,%
            当值为“01”时选取imme,当值为“10”时选取cp0\_value,%
            当值为“11”时选取立即数16。需要在clk时钟上升沿之前准备好,%
            并保持到clk时钟上升沿之后一极短时间。
            } \\
            \midrule
            alu\_srcB & in & std\_logic\_vector(1 downto 0) \\
            \cmidrule(l){2-3}
            &
            \multicolumn{2}{X}{
            ALU的第二个操作数的选择码,当值为“00”时选取rt\_value,%
            当值为“01”时选取imme,当值为“10”时选取cp0\_value。%
            需要在clk时钟上升沿之前准备好,%
            并保持到clk时钟上升沿之后一极短时间。
            } \\
            \midrule
            alu\_result & out & std\_logic\_vector(31 downto 0) \\
            \cmidrule(l){2-3}
            &
            \multicolumn{2}{X}{
            ALU的输出,在下一个时钟上升沿之前准备好,%
            并保持直到下一次能使ALU输出改变%
            (state为ALU工作状态、alu\_op非写hi、lo寄存器)%
            的时钟上升沿。
            } \\
            \bottomrule
            \end{tabularx}
