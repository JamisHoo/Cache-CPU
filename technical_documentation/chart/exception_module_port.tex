        \begin{tabularx}{\textwidth}{lll}
            \toprule
            端口名          & 端口方向  & 端口类型 \\
            \cmidrule(l){2-3}
            &
            \multicolumn{2}{X}{端口描述} \\
            \midrule
            clk             & in        & std\_logic \\
            \cmidrule(l){2-3}
            &
            \multicolumn{2}{X}{
                CPU时钟信号
            } \\
            \midrule
            state           & in        & status \\
            \cmidrule(l){2-3}
            &
            \multicolumn{2}{X}{
                自定义状态集合
            } \\
            \midrule
            exception\_e    & in        & std\_logic \\
            \cmidrule(l){2-3}
            &
            \multicolumn{2}{X}{
                exception模块使能信号
            } \\
            \midrule
            mmu\_exc\_code  & in        & std\_logic\_vector(2 downto 0) \\
            \cmidrule(l){2-3}
            &
            \multicolumn{2}{X}{
                来自MMU的异常信号,%
                表示TLB\_MODIFIED、TLB\_L、TLB\_S、ADE\_L、ADE\_S异常。%
                要求exception阶段时钟上升沿之前保持。
            } \\
            \midrule
            serial\_int     & in        & std\_logic \\
            \cmidrule(l){2-3}
            &
            \multicolumn{2}{X}{
                来自串口的异常信号。要求exception阶段时钟上升沿之前保持。
            } \\
            \midrule
            compare\_interrupt & in     & std\_logic \\
            \cmidrule(l){2-3}
            &
            \multicolumn{2}{X}{
                来自CP0的时钟中断信号。要求exception阶段时钟上升沿之前保持。
            } \\
            \midrule
            id\_exc\_code   & in        & std\_logic\_vetor(1 downto 0) \\
            \cmidrule(l){2-3}
            &
            \multicolumn{2}{X}{
                来自ID的异常信号,表示SYSCAL,RI异常。要求exception阶段时钟上升沿之前保持。
            } \\
            \midrule
            pc\_in          & in        & std\_logic\_vector(31 downto 0) \\
            \cmidrule(l){2-3}
            &
            \multicolumn{2}{X}{
                本指令的PC,来自CPU模块。要求exception阶段时钟上升沿之前保持。
            } \\
            \midrule
            v\_addr\_in     & in        & std\_logic\_vector(31 downto 0) \\
            \cmidrule(l){2-3}
            &
            \multicolumn{2}{X}{
                目前的访存虚拟地址,来自MMU。%
                要求exception阶段时钟上升沿之前保持。
            } \\
            \midrule
	        old\_entry\_hi  & in        & std\_logic\_vector(19 downto 0) \\
            \cmidrule(l){2-3}
            &
            \multicolumn{2}{X}{
                旧的entry\_hi,用于entry\_hi不变的情况,来自CP0。%
                要求exception阶段时钟上升沿之前保持。
            } \\
            \midrule
            old\_interrupt\_code & in   & std\_logic\_vector(5 downto 0) \\
            \cmidrule(l){2-3}
            &
            \multicolumn{2}{X}{
                旧的中断号,用于中断号不变的情况,来自CP0。%
                要求exception阶段时钟上升沿之前保持。
            } \\
            \midrule
            bad\_v\_addr\_out & out     & std\_logic\_vector(31 downto 0) \\
            \cmidrule(l){2-3}
            &
            \multicolumn{2}{X}{
                bad\_v\_addr输出值,交给CP0模块进行写入,可以保持到下次改变。
            } \\
            \midrule
            entry\_hi\_out  & out       & std\_logic\_vector(19 downto 0) \\
            \cmidrule(l){2-3}
            &
            \multicolumn{2}{X}{
                entry\_hi输出值,交给CP0模块进行写入,可以保持到下次改变。
            } \\
            \midrule
	        interrupt\_start\_out & out & std\_logic \\
            \cmidrule(l){2-3}
            &
            \multicolumn{2}{X}{
                CP0模块的异常写入使能,%
                控制CP0模块开始写入异常信息,%
                可以保持到下一时钟上升沿之前。
            } \\
            \midrule
	        cause\_out      & out       & std\_logic\_vector(4 downto 0) \\
            \cmidrule(l){2-3}
            &
            \multicolumn{2}{X}{
                异常号输出值,交给CP0模块进行写入,可以保持到下次改变。
            } \\
            \midrule
	        interrupt\_cause\_out & out & std\_logic\_vector(5 downto 0) \\
            \cmidrule(l){2-3}
            &
            \multicolumn{2}{X}{
                中断号输出值,交给CP0模块进行写入,可以保持到下次改变。
            } \\
            \midrule
	        epc\_out        & out       & std\_logic\_vector(31 downto 0) \\
            \cmidrule(l){2-3}
            &
            \multicolumn{2}{X}{
                EPC输出值,交给CP0模块进行写入,可以保持到下次改变
            } \\
            \midrule
	        pc\_sel0        & out       & std\_logic \\
            \cmidrule(l){2-3}
            &
            \multicolumn{2}{X}{
                IF阶段选择PC的pc\_sel的0位,%
                若为1表示应选择异常处理向量作为新的PC。%
                可以保持到下一时钟上升沿之前。
            } \\

            \bottomrule
        \end{tabularx}


