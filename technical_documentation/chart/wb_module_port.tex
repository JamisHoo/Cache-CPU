        \begin{tabularx}{\textwidth}{lll}
            \toprule
            端口名          & 端口方向  & 端口类型 \\
            \cmidrule(l){2-3}
            &
            \multicolumn{2}{X}{端口描述} \\
            \midrule
            clk             & in        & std\_logic \\
            \cmidrule(l){2-3}
            &
            \multicolumn{2}{X}{
                CPU时钟信号
            } \\
            \midrule
            state           & in        & status(自定义状态集合) \\
            \cmidrule(l){2-3}
            &
            \multicolumn{2}{X}{
                CPU当前状态
            } \\
            \midrule
            WB\_e           & in        & std\_logic \\
            \cmidrule(l){2-3}
            &
            \multicolumn{2}{X}{
                WB文件的使能信号
            } \\
            \midrule
            RPC             & in        & std\_logic\_vector(31 downto 0) \\
            \cmidrule(l){2-3}
            &
            \multicolumn{2}{X}{
                即本周期的PC+4,要求ALU阶段上升沿之前准备好。
            } \\
            \midrule
            mmu\_value      & in        & std\_logic\_vector(31 downto 0) \\
            \cmidrule(l){2-3}
            &
            \multicolumn{2}{X}{
                来自MMU的读取值,要求WB阶段上升沿之前准备好。
            } \\
            \midrule
            cp0\_value      & in        & std\_logic\_vector(31 downto 0) \\
            \cmidrule(l){2-3}
            &
            \multicolumn{2}{X}{
                来自CP0的读取值,要求WB阶段上升沿之前准备好。
            } \\
            \midrule
            alu\_result     & in        & std\_logic\_vector(31 downto 0) \\
            \cmidrule(l){2-3}
            &
            \multicolumn{2}{X}{
                来自ALU的读取值,要求WB阶段上升沿之前准备好。
            } \\
            \midrule
            wb\_op          & in        & std\_logic\_vector(4 downto 0) \\
            \cmidrule(l){2-3}
            &
            \multicolumn{2}{X}{
                WB阶段写入寄存器编号的来源与写入数据的来源选择,%
                要求WB阶段上升沿之前准备好。

                4-3位为写入寄存器编号来源的控制信号:

                    00表示写入rt寄存器;

                    01表示写入rd寄存器;

                    10表示写入31寄存器;

                    11或其他信号表示不写;
            } \\
            &
            \multicolumn{2}{X}{
                2-0位为写入数据来源的控制信号:

                    000表示数据来自ALU;

                    001表示数据来自MMU;

                    010表示数据来自RPC;

                    011表示数据来自MMU,此时为LBU操作,%
                    根据alu\_result的低2位决定使用MMU数据的哪一字节,%
                    高位零扩展;
            } \\
            &
            \multicolumn{2}{X}{
                    100表示数据来自MMU,此时为LB操作,%
                    根据alu\_result的低2位决定使用MMU数据的哪一字节,%
                    高位符号扩展;

                    101表示数据来自MMU,此时为LHU操作,%
                    根据alu\_result的低2位决定使用MMU数据的哪一字节,%
                    高位零扩展;

                    110表示数据来自CP0寄存器;
            } \\
            \midrule
            rd\_addr        & in    & std\_logic\_vector(4 downto 0) \\
            \cmidrule(l){2-3}
            &
            \multicolumn{2}{X}{
                rd寄存器编号,要求WB阶段上升沿之前准备好。
            } \\
            \midrule
            rt\_addr        & in    & std\_logic\_vector(4 downto 0) \\
            \cmidrule(l){2-3}
            &
            \multicolumn{2}{X}{
                rt寄存器编号,要求WB阶段上升沿之前准备好。
            } \\
            \midrule
            write\_addr     & out   & std\_logic\_vector(4 downto 0) \\
            \cmidrule(l){2-3}
            &
            \multicolumn{2}{X}{
                写入通用寄存器组的地址,请在每一时钟上升沿使用。%
                保持到下一WB阶段时钟上升沿。

                对于IP core的片内RAM模块组成的通用寄存器组,%
                此信号请直接接至通用寄存器组。
            } \\
            \midrule
            write\_value    & out   & std\_logic\_vector(31 downto 0) \\
            \cmidrule(l){2-3}
            &
            \multicolumn{2}{X}{
                写入通用寄存器组的数据,请在每一时钟上升沿使用。%
                保持到下一WB阶段时钟上升沿。

                对于IP core的片内RAM模块组成的通用寄存器组,%
                此信号请直接接至通用寄存器组。
            } \\
            \midrule
            write\_enable   & out   & std\_logic \\
            \cmidrule(l){2-3}
            &
            \multicolumn{2}{X}{
                写入通用寄存器组的使能,请在每一时钟上升沿使用。%
                保持到下一WB阶段时钟上升沿。

                对于IP core的片内RAM模块组成的通用寄存器组,%
                此信号请直接接至通用寄存器组。
            } \\
            \midrule
	        pc\_op          & in    & std\_logic\_vector(2 downto 0) \\
            \cmidrule(l){2-3}
            &
            \multicolumn{2}{X}{
                非ERET指令时,选择新PC的控制信号                                             

                    00时,选PC+4,即RPC;                                                    

                    01时,根据比较是否成立进行选择,%
                    成立则跳转到PC+4+immediate,即RPC+immediate;否则跳转到RPC。
            } \\
            &
            \multicolumn{2}{X}{
                    10时,跳转到immediate << 2,对应某些J、JAL语句;

                    11时,跳转到ALU的计算结果,对应某些JALR、JR语句;

                要求WB阶段上升沿之前准备好。
            } \\
            \midrule
            comp\_op        & in    & std\_logic\_vector(2 downto 0) \\
            \cmidrule(l){2-3}
            &
            \multicolumn{2}{X}{
                比较跳转时的比较控制信号,一般对应B系列指令。跳转条件如下:

                    000时,条件为rs的值等于rt的值;

                    001时,条件为rs的值>=0;

                    010时,条件为rs的值>0;
            } \\
            &
            \multicolumn{2}{X}{
                    011时,条件为rs的值<=0;

                    100时,条件为rs的值<0;

                    101时,条件为rs的值不等于rt的值;

                    其他情况,恒为非;

                要求ALU阶段上升沿之前准备好。
            } \\
            \midrule
            rs\_value       & in    & std\_logic\_vector(31 downto 0) \\
            \cmidrule(l){2-3}
            &
            \multicolumn{2}{X}{
                rs的值,要求ALU阶段上升沿之前准备好。
            } \\
            \midrule
            rt\_value       & in    & std\_logic\_vector(31 downto 0) \\
            \cmidrule(l){2-3}
            &
            \multicolumn{2}{X}{
                rt的值,要求ALU阶段上升沿之前准备好。
            } \\
            \midrule
            imme            & in    & std\_logic\_vector(31 downto 0) \\
            \cmidrule(l){2-3}
            &
            \multicolumn{2}{X}{
                指令中包含的立即数,要求WB阶段上升沿之前准备好。
            } \\
            \midrule
            PcSrc            & out   & std\_logic\_vector(31 downto 0) \\
            \cmidrule(l){2-3}
            &
            \multicolumn{2}{X}{
                非ERET指令时,新PC值。                                                       
                WB阶段上升沿之后可读。
            } \\
                

            \bottomrule
        \end{tabularx}
