        \begin{tabularx}{\textwidth}{lll}
            \toprule
            端口名          & 端口方向  & 端口类型 \\
            \cmidrule(l){2-3}
            &
            \multicolumn{2}{X}{端口描述} \\
            \midrule
            clk             & in        & std\_logic \\
            \cmidrule(l){2-3}
            &
            \multicolumn{2}{X}{
                CPU时钟信号。
            } \\
            \midrule
            state           & in        & std\_logic\_vector(3 downto 0) \\
            \cmidrule(l){2-3}
            &
            \multicolumn{2}{X}{
                CPU当前状态。
            } \\
            \midrule
            instruction     & in        & std\_logic\_vector(31 downto 0) \\
            \cmidrule(l){2-3}
            &
            \multicolumn{2}{X}{
                说明:当前指令。

                来源:MMU模块。

                到达时间:当前指令InsD上升沿之前。

                产生时间:当前指令InsF上升沿之后。
            } \\
            \midrule
            instr\_out      & out       & std\_logic\_vector(31 downto 0) \\
            \cmidrule(l){2-3}
            &
            \multicolumn{2}{X}{
                说明:指令寄存器,除三个寄存器的编号,%
                其他所有控制线均从此产生。%
                之后周期中如果需要用到指令也从此处获得。

                产生时间:当前指令InsD上升沿之后。
            } \\
            \midrule
            rs\_addr        & out       & std\_logic\_vector(4 downto 0) \\
            \cmidrule(l){2-3}
            &
            \multicolumn{2}{X}{
                说明:通用寄存器编号1,在InsD阶段需要读取到值。

                产生时间:当前指令InsD上升沿之前。
            } \\
            \midrule
            rt\_addr        & out       & std\_logic\_vector(4 downto 0) \\
            \cmidrule(l){2-3}
            &
            \multicolumn{2}{X}{
                说明:通用寄存器编号2、写入寄存器编号,在InsD阶段需要读取到值。

                产生时间:当前指令InsD上升沿之前。
            } \\
            \midrule
            rd\_addr        & out       & std\_logic\_vector(4 downto 0) \\
            \cmidrule(l){2-3}
            &
            \multicolumn{2}{X}{
                说明:CP0寄存器编号、写入寄存器编号,在InsD阶段需要读取到值。

                产生时间:当前指令InsD上升沿之前。
            } \\
            \midrule
            pc\_op          & out       & std\_logic\_vector(1 downto 0) \\
            \cmidrule(l){2-3}
            &
            \multicolumn{2}{X}{
                说明:PCSrc选择器,正常状态下PC的选择方式。%
                输出到WB模块,4选1数据选择器的控制信号,选择正确的PC。

                产生时间:当前指令InsD上升沿之后。

                有效时间:下一条指令InsD上升沿之后。 
            } \\
            &
            \multicolumn{2}{X}{
                00 PC+4的计算结果。

                01 branch指令PC计算的结果。

                10 jump指令目标地址。

                11 主ALUOut计算结果。
            } \\
            \midrule
            eret\_enable    & out       & std\_logic \\

            \cmidrule(l){2-3}
            &
            \multicolumn{2}{X}{
                说明:ERET使能,对PC进行选择。%
                专门对ERET指令使用,输出到IFetch模块。

                产生时间:当前指令InsD上升沿之后。

                有效时间:下一条指令InsD上升沿之后。 
            } \\
            \midrule
            comp\_op        & out       & std\_logic\_vector(2 downto 0) \\
            \cmidrule(l){2-3}
            &
            \multicolumn{2}{X}{
                说明:比较信号,branch指令的跳转条件。%
                输出到WB模块,如果为branch系列指令则通过此信号进行选择。

                产生时间:当前指令InsD上升沿之后。

                有效时间:下一条指令InsD上升沿之后。 
            } \\
            &
            \multicolumn{2}{X}{
                000 BEQ(R[s] = R[t])

                001 BGEZ(R[s] >= 0)

                010 BGTZ(R[s] > 0)

                011 BLEZ(R[s] <= 0)

                100 BLTZ(R[s] < 0)

                101 BNE(R[s] != R[t])
            } \\
            \midrule
            imme            & out       & std\_logic\_vector(31 downto 0) \\
            \cmidrule(l){2-3}
            &
            \multicolumn{2}{X}{
                说明:32位立即数,针对不同指令的需求产生。%
                立即数本身作为ALUSrc的来源之一。

                产生时间:当前指令InsD上升沿之后。

                有效时间:下一条指令InsD上升沿之后。 
            } \\
            &
            \multicolumn{2}{X}{
                立即数产生方式有四种,均为从不同立即数长度扩展为32位。

                16位有符号扩展,16位无符号扩展,%
                移位指令需要的5位到32位扩展,jump指令的立即数扩展。
            } \\
            \midrule
            alu\_ops        & out       & std\_logic\_vector(8 downto 0) \\
            \cmidrule(l){2-3}
            &
            \multicolumn{2}{X}{
                说明:控制ALU模块。

                ALUSrcA(8 downto 7):ALU第一输入的选择信号,%
                四选一数据选择。

                ALUSrcB(6 downto 5):ALU第二输入的选择信号,%
                四选一数据选择。

                ALUOp(4 downto 0):ALU操作,%
                5位,整合了乘法器相关运算,详细说明见ALU模块说明。

                产生时间:当前指令InsD上升沿之后。

                有效时间:下一条指令InsD上升沿之后。 
            } \\
            \midrule
            mem\_op         & out       & std\_logic\_vector(2 downto 0) \\
            \cmidrule(l){2-3}
            &
            \multicolumn{2}{X}{
                说明:控制MEM模块。

                MEMRead(2):是否可读内存。

                MEMWrite(1):是否可写内存。

                MEMValue(0):选择写入内存的值,%
                寄存器的数据或者SB指令处理之后的数据。

                产生时间:当前指令InsD上升沿之后。

                有效时间:下一条指令InsD上升沿之后。
            } \\
            \midrule
            wb\_op          & out       & std\_logic\_vector(5 downto 0) \\
            \cmidrule(l){2-3}
            &
            \multicolumn{2}{X}{
                说明:控制WB模块。

                产生时间:当前指令InsD上升沿之后。

                有效时间:下一条指令InsD上升沿之后。 

                RegDst(5 downto 4):写回寄存器编号。

                如果为00则写回16到20位rt寄存器,%
                如果为01则写回11到15位rd寄存器,%
                如果为10则写回31号寄存器。

                RegWrite(0):寄存器是否可写。
            } \\
            &
            \multicolumn{2}{X}{
                RegValue(3 downto 1):写回寄存器的内容%

                000 主ALU计算结果

                001 读取内存的数据

                010 RPC

                011 Zero-extend内存byte

                100 Signed-extend内存byte

                101 Zero-extend内存word

                110 CP0寄存器
            } \\
            \midrule
            cp0\_op         & out       & std\_logic\_vector(1 downto 0) \\
            \cmidrule(l){2-3}
            &
            \multicolumn{2}{X}{
                说明:控制CP0模块。

                EPCValue(1):异常产生时,EPC写入的内容,%
                选择写入PC或者PC+4。

                CP0Write(0):CP0寄存器是否可写。

                产生时间:当前指令InsD上升沿之后。

                有效时间:下一条指令InsD上升沿之后 
            } \\
            \midrule
            tlbwi\_enable   & out       & std\_logic \\
            \cmidrule(l){2-3}
            &
            \multicolumn{2}{X}{
                说明:TLB写使能。

                产生时间:当前指令InsD上升沿之后。

                有效时间:下一条指令InsD上升沿之后。
            } \\
            \bottomrule
            \end{tabularx}
