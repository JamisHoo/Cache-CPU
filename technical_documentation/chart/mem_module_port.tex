        \begin{tabularx}{\textwidth}{lll}
            \toprule
            端口名      & 端口方向  & 端口类型 \\
            \cmidrule(l){2-3}
            &
            \multicolumn{2}{X}{端口描述} \\
            \midrule
            result      & in   & std\_logic\_vector(31 downto 0) \\
            \cmidrule(l){2-3}
            &
            \multicolumn{2}{X}{
                说明:访存地址

                来源:ALU模块

                到达时间:指令执行阶段时钟上升沿后。

                保持时间:至少保持到访存阶段时钟上升沿之后。
            } \\
            \midrule
            rst      & in    & std\_logic \\
            \cmidrule(l){2-3}
            &
            \multicolumn{2}{X}{
                初始化信号,在CPU启动时使用。
            } \\
            \midrule
            mem\_op      & in    & std\_logic\_vector(2 downto 0) \\
            \cmidrule(l){2-3}
            &
            \multicolumn{2}{X}{
                说明:内存操作控制线

                来源:IDecode模块
                
                到达时间:指令解码时钟上升沿之后。

                保持时间:至少保持到访存阶段结束。
            } \\
            &
            \multicolumn{2}{X}{
                mem\_op(2): memRead:内存读使能。

                mem\_op(1): memWrite:内存写使能。

                mem\_op(0): memValue:内存写入数据。
                如果为0则写入来自寄存器的数据,%
                如果为1则写入SB指令处理之后的数据。
            } \\
            \midrule
            rt\_value   & in    & std\_logic\_vector(31 downto 0) \\
            \cmidrule(l){2-3}
            &
            \multicolumn{2}{X}{
                说明:来自寄存器的数值,访存阶段写入数据的可能来源之一。

                来源:通用寄存器。

                到达时间:指令解码时钟上升沿之后。
            } \\
            \midrule
            mmu\_value  & in    & std\_logic\_vector(31 downto 0) \\
            \cmidrule(l){2-3}
            &
            \multicolumn{2}{X}{
                说明:访存得到的数据,为SB指令提供支持。

                来源:MMU模块。

                到达时间:第二次访存操作上升沿之前。

                产生时间:第一次访存操作下降沿之后。
            } \\
            \midrule
            addr\_mmu    & out  & std\_logic\_vector(31 downto 0) \\
            \cmidrule(l){2-3}
            &
            \multicolumn{2}{X}{
                说明:访存阶段的地址。

                产生时间:访存阶段时钟上升沿之前。
            } \\
            \midrule
            write\_value & out & std\_logic\_vector(31 downto 0) \\
            \cmidrule(l){2-3}
            &
            \multicolumn{2}{X}{
                说明:访存阶段写入数据。

                产生时间:访存阶段时钟上升沿之前。
            } \\
            \midrule
            read\_enable & out & std\_logic \\
            \cmidrule(l){2-3}
            &
            \multicolumn{2}{X}{
                说明:访存读使能。
            } \\
            \midrule
            write\_enable & out & std\_logic \\
            \cmidrule(l){2-3}
            &
            \multicolumn{2}{X}{
                说明:访存写使能。
            } \\
            \bottomrule
        \end{tabularx}
