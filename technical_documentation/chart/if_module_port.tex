        \begin{tabularx}{\textwidth}{lll}
            \toprule
            端口名          & 端口方向  & 端口类型 \\
            \cmidrule(l){2-3}
            &
            \multicolumn{2}{X}{端口描述} \\
            \midrule
            clk             & in        & std\_logic \\
            \cmidrule(l){2-3}
            &
            \multicolumn{2}{X}{
                CPU时钟信号
            } \\
            \midrule
            rst             & in        & std\_logic \\
            \cmidrule(l){2-3}
            &
            \multicolumn{2}{X}{
                初始化信号,在CPU启动时使用。
            } \\
            \midrule
            state           & in        & status(自定义状态集合) \\
            \cmidrule(l){2-3}
            &
            \multicolumn{2}{X}{
                CPU当前状态
            } \\
            \midrule
            mmu\_ready           & in        & std\_logic \\
            \cmidrule(l){2-3}
            &
            \multicolumn{2}{X}{
                标志访存是否结束
            } \\
            \midrule
            PCSrc           & in        & std\_logic\_vector(31 downto 0) \\
            \cmidrule(l){2-3}
            &
            \multicolumn{2}{X}{
                说明:非异常状态下的指令地址。

                来源:WB模块。

                到达时间:当前指令InsF上升沿之前。

                产生时间:上一条指令WB上升沿之后。
            } \\
            \midrule
            EBase           & in        & std\_logic\_vector(31 downto 0) \\
            \cmidrule(l){2-3}
            &
            \multicolumn{2}{X}{
                说明:异常处理基地址。

                来源:CP0模块。

                到达时间:当前指令InsF上升沿到来之前。
            } \\
            \midrule
            EPC             & in        & std\_logic\_vector(31 downto 0) \\
            \cmidrule(l){2-3}
            &
            \multicolumn{2}{X}{
                说明:ERET指令的返回地址。

                来源:CP0模块。

                到达时间:当前指令InsF上升沿到来之前。
            } \\
            \midrule
            pc\_sel         & in        & std\_logic\_vector(1 downto 0) \\
            \cmidrule(l){2-3}
            &
            \multicolumn{2}{X}{
                pc\_sel(1):

                    说明:eret\_enable,使能信号。

                    来源:ID模块。

                    产生时间:上一条指令ID上升沿之后。

                    持续时间:直到下一条指令的IF阶段。
            } \\
            &
            \multicolumn{2}{X}{
                pc\_sel(0):

                    说明:pc\_control,判断是否为异常状态。

                    来源:异常模块。

                    到达时间:当前指令InsF上升沿到来之前。
            } \\
            \midrule
            PC              & out       & std\_logic\_vector(31 downto 0) \\
            \cmidrule(l){2-3}
            &
            \multicolumn{2}{X}{
                说明:PC寄存器,时序逻辑。

                产生时间:当前指令InsF上升沿到来之后。

                有效时间:直到下一条指令的IF阶段。
            } \\
            \midrule
            PCmmu           & out       & std\_logic\_vector(31 downto 0) \\
            \cmidrule(l){2-3}
            &
            \multicolumn{2}{X}{
                说明:为MMU单元提供的PC,组合逻辑。

                产生时间:当前指令InsF上升沿到来之前。

                有效时间:直到当前指令IF阶段结束。
            } \\
            \bottomrule
        \end{tabularx}
