\section{简介}
    本实验主要目的是修改Decaf/Mind编译器,%
    使之能够在简单的MIPS32指令系统处理器以及操作系统$\mu$core上运行。
    \subsection*{实验环境}
    \begin{enumerate}
        \item
        支持MIPS32指令子集的多周期CPU,主频6.25MHz。
        \item
        可在上述CPU上运行的类Unix操作系统$\mu$core。
        \item
        Decaf/Mind编译器,可以将decaf程序源代码编译成MIPS32汇编代码。
    \end{enumerate}

\section{实现方法}
    \subsection{指令集}
        现有Decaf编译器生成的部分指令不在CPU支持范围内。%
        这些指令和相应的解决方法如下:
        \begin{enumerate}
        \item
        add、sub、neg
            
        将这三条生成的指令分别修改为addu、subu、negu。%
        区别在于修改后的指令在算术溢出时不会产生异常,%
        不会对正常运行产生影响。

        在Mips类genAsmForBB函数中将add修改成addu,%
        将sub修改成subu,将neg修改成negu。
        \item
        mul

        将mul指令拆分成mult和mflo两条指令,%
        拆分后指令与原指令等价。%
        需要在翻译、TAC生成、基本块、寄存器分配、%
        汇编代码生成这5个部分修改Decaf编译器。

        在Translater类genMul函数中将一条mul修改成两条指令,%
        在Tac类Kind枚举中加入MULT、MFLO,%
        在Tac类中加入成员函数genMult、genMflo,%
        在BasicBlock类computeDefAndLiveUse函数中加入对MULT和MFLO的处理,%
        在RegisterAllocator类alloc函数中加入对MULT和MFLO的处理,%
        在Mips类genAsmForBB函数中增加对MULT和MFLO的处理。

        \item
        div、rem

        使用加减交替法将除法替换成加减算术运算、移位运算和逻辑运算。%
        将加减交替法实现成一个库函数。%
        修改Decaf编译器使之将除法、模运算编译成函数调用。%
        需要在翻译、库函数调用两个部分修改Decaf编译器。

        在TransPass2类visitBinary函数中将DIV和REM修改成传递两个参数的函数调用,%
        在Intrinsic类中增加对decaf\_Div和decaf\_Rem函数的处理。
        \end{enumerate}

    \subsection{函数调用约定}
        $\mu$core操作系统遵循o32ABI标准,%
        函数调用的前16字节参数通过寄存器a0$\sim$a3传递。%
        Decaf编译器得到的程序所有参数都通过栈传递。%
        对于Decaf编译得到的内部函数之间的调用,%
        程序可以正常运行,但对库函数则不能正常调用。

        考虑到本次实验所需要实现的库函数参数均未超过16字节,%
        而且a0$\sim$a3是调用者保存寄存器,%
        可以使用汇编语言在调用函数和被调用函数之间添加一层调用,%
        将栈上保存的参数储存到a0$\sim$a3寄存器中,%
        例如将第一个参数从sp+4复制到a0寄存器,%
        不需要对栈帧指针等其他内容做任何维护。

    \subsection{库函数}
        使用C语言实现PrintString、PrintInt、PrintBool、%
        ReadLine、ReadInteger、Halt、Allocate、除法、模运算%
        共计9个库函数。%
        按照前文提到的库函数调用方法,%
        将Decaf编译器产生的汇编程序与库函数链接即可得到二进制文件。
        
        在Intrinsic类中将函数名称改成对带有decaf\_前缀的中间层函数调用,%
        在decaf\_entry.S中将位于栈上的参数复制到参数寄存器中,%
        并通过j指令直接跳转到目标库函数入口,%
        不会对栈帧寄存器、ra寄存器产生任何影响。%
        目标库函数实现在intrinsic.c中。%

    \subsection{程序退出}
        $\mu$core操作系统默认程序退出应当返回0,否则按出错处理。%
        Decaf语言main函数为void类型,不返回值。%
        处理方法是在main函数退出前添加一条指令move \$v0, \$zero,%
        需要修改汇编代码生成部分。

        在Mips类emitTrace函数中增加判断,%
        如果是函数名为main则在函数退出恢复栈帧指针时将v0寄存器清零。

    \subsection{string类型}
        Decaf语言中string类型存储字符串的首指针,%
        从标准输入读取string类型实际上是%
        将含有字符串的缓冲区的首指针赋值给string变量。%
        另外Decaf语言没有垃圾回收机制,动态分配的空间无法释放。%
        由此造成的直接后果是操作系统不知道何时应该释放string的存储空间,%
        在每次从标准输入读取string时必须分配新的缓冲区并保留直至程序退出。


\section{实验结果}
    修改后的Decaf编译器能够正确运行含有乘法、除法运算的math程序,%
    能够正确运行含有递归函数调用的fibonacci程序,%
    能够正确运行含有标准输入输出、模运算的blackjack程序。

\section{人员分工}
    在本次实验中,%
    胡津铭负责对Decaf/Mind编译器的修改,%
    李天润负责编写C语言库函数,%
    孙皓负责库编写库函数调用中间层函数。

    所有添加、修改已在源代码中标注。
