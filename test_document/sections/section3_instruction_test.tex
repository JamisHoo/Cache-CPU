\section{单指令测试}

    \subsection{算数指令}

        \subsubsection{测试方法}
            算数指令是最基本的指令,因此放在最前面测试。

            算数指令大多会写会通用寄存器组,由于之前已经确定通用寄存器组可以正确工作,此处只需检查给到通用寄存器组的地址、数值、写信号是否正确即可。

            由于此时整个CPU的整合尚未完全完成,不能确定访存路径是否正确,通过拨码开关代替访存部分,即将所有需要从访存取得的数据改由拨码开关输入%
            ,同时手动按时钟进行输入。通过LED灯检查给通用寄存器组的写信号是否符合预期。

            另外,随着测试的进行,调试工具也在同步开发。在指令测试的后半部分时,加入调试工具,在电脑端查看输出信号。

        \subsubsection{测试样例}
            由于开始时所有通用寄存器的数值均为0,需要给入恰当的数据才能有效检查,因此按照由易入难的顺序进行测试。
            \begin{enumerate}
            \item
                ADDIU \\
                ADDIU 1 0 4 \\
                ADDIU 2 1 -1
            \item
                ADDU \\
                ADDIU 1 0 4 \\
                ADDIU 2 0 -2 \\
                ADDU 3 1 2
            \item
                SLT \\
                ADDIU 1 0 4 \\
                ADDIU 3 0 5 \\
                SLT 2 1 3 \\
                SLT 2 3 1 \\
                ADDIU 3 0 -2 \\
                SLT 2 1 3 \\
                SLT 2 3 1 \\
            \item
                SLTI \\
                ADDIU 1 0 4 \\
                SLTI 3 1 3 \\
                SLTI 3 1 5 \\
                SLTI 3 1 -1
            \item
                SLTIU \\
                ADDIU 1 0 4 \\
                SLTIU 3 1 3 \\
                SLTIU 3 1 5 \\
                SLTIU 3 1 -1
            \item
                SLTU \\
                ADDIU 1 0 4 \\
                ADDIU 3 0 5 \\
                SLTU 2 1 3 \\
                SLTU 2 3 1 \\
                ADDIU 3 0 -2 \\
                SLTU 2 1 3 \\
                SLTU 2 3 1 \\
            \item
                SUBU \\
                ADDIU 1 0 4 \\
                ADDIU 2 0 -2 \\
                SUBU 3 1 2
            \item
                MULT,MFLO,MFHI\\
                ADDIU 1 0 4 \\
                ADDIU 3 0 5 \\
                MULT 1 3 \\
                MFLO 2 \\
                MFHI 2
            \item
                MTLO,MTHI \\
                ADDIU 1 0 4 \\
                ADDIU 3 0 5 \\
                MTLO 1 \\
                MTHI 3 \\
                MFLO 2 \\
                MFHI 2
            \item
                AND,ANDI,LUI,NOR,OR,ORI,XOR,XORI,SLL,SLLV,SRA,SRAV,SRL,SRLV,LHU采用类似的方法进行测试
            \end{enumerate}


        \subsubsection{问题与解决}
        测试的过程中发现了一系列错误,包括写会通用寄存器的操作控制信号设计有误,写地址来源在一种控制信号组合下有误等。

        通过检查给到通用寄存器的信号,可以判断是否发生错误。发生错误后,要逆着数据通路检查沿途的控制信号、数据信号。%
        这个操作一开始通过LED灯进行检查,之后加入调试工具,可以直接在每次时钟之后看到所有沿途信号,进行检查。
        最终,以上指令均检查完成

    \subsection{跳转指令}
        \subsubsection{测试方法}
        跳转指令除了通用寄存器,还需要检查PC的变化,以及RPC的保存。

        \subsubsection{测试说明}
        跳转指令包括BEQ,BGEZ,BGTZ,BLEZ,BLTZ,BNE,J,JAL,JALR,JR,ERET \\
        检查时,需要对正、负值都进行检查。

        \subsubsection{测试结果}
        确认以上指令可以按照实验指导书的定义运行。\\
        另外,之后运行ucore时发现实验指导书的定义是错误的,尤其是J类指令的左移两位问题,在确定新的要求后进行了修改。

    \subsection{访存指令}
        \subsubsection{测试方法}
        预先通过controller.py在内存与Flash中写入一段数据,%
        之后通过load与store指令进行数据的拷贝、覆盖等操作,%
        最后通过controller.py将结果读出与输入做比较。

        \subsubsection{测试说明}
        物理访存模块结合MMU模块共同测试,%
        需要测试的访存指令包括LW、SW、LB、LBU、SB、LHU。%
        按word进行操作的指令,在RAM、Flash、串口均进行测试,%
        按half-word与byte进行操作的指令,仅对RAM做测试。

        \subsubsection{测试结果}
        测试中发现访存指令工作不稳定,%
        表现为访存指令有一定概率错误地触发异常,%
        或者读取出错误的数据。%
        可能的原因是该阶段采用的手动时钟,%
        在按下开关的瞬间可能产生毛刺,%
        导致访存指令出现多次访存、锁存地址错误等情况。%
        后将该测试移至指令片段测试阶段,%
        在高速时钟下,该问题得到解决。

    \subsection{特殊指令}
        此部分包括SYSCALL,CACHE,NOP,TLBWI,采用与其他指令相同的方法进行输入,然后通过对控制信号的监控进行检查。
