\section{单元测试}

    %感觉有些模块在这个阶段没什么可说的
    %比如取指令模块等等,虽然最后发现了问题,
    %但是并不是在单元测试阶段可以发现的
    %所以暂时只列出一下几个模块

    \subsection{ALU模块}

        % \subsubsection{功能概述}
        %    简要说明,需要达到怎样的功能
        % 述个屁啊,让他自己到需求文档去看。

        \subsubsection{测试方法}
            通过拨码开关输入操作数、操作符检查运算结果正确性。%
            使用状态机,在高频下自动产生操作数、操作符,%
            并于VHDL内置运算结果比较,以检验在高频下连续运算结果的正确性。%
            除此之外,还需要检查高频下乘法速度是否满足需要。

        \subsubsection{测试用例}
            除了简单测例、边界测例外,还需要测试加、减、移位运算的溢出。
            测例诸如%
            0xffffffff+1、%
            0xffffffff+2、%
            0x00000001-2等。

        \subsubsection{测试结果}
            所有运算操作结果正确,乘法运算可以在25MHz频率下一个周期内得出正确结果。

    \subsection{指令解码模块}
        \subsubsection{功能概述}
            在指令解码周期的时钟上升沿,%
            产生该指令所需的全部控制信号,%
            并对当前指令、通用寄存器编号进行锁存。

        \subsubsection{测试方法}
            通过拨码开关输入指令,%
            状态机始终控制在解码阶段,%
            通过手动时钟提供时钟上升沿,%
            将产生的控制信号输出到LED灯查看。%

        \subsubsection{测试用例}
            使用全部MIPS32指令进行测试。%
            但是因为实验板上LED灯只有16个,%
            一次只能查看部分信号,%
            且输入指令占用全部的32个拨码开关,%
            导致测试效果不佳。%

            因此在该阶段只进行了少量测试,%
            指令解码模块的测试工作后延至单指令测试阶段。

        \subsubsection{测试结果}
            指令解码模块在测试中发现了如下问题:
            \begin{enumerate}
            \item
                JAL和J指令的定义在原有指导文档上有误,%
                地址的高四位并不发生改变。%
                上网查找了标准的MIPS32指令进行修正,%
                之后JAL和J指令能够正常工作。
            \item
                ALU控制信号ALUOp错误。%
                编码过程中曾对ALUOp进行过一次修改,%
                但并没有体现在数据通路表中。%
                修改了数据通路表和相关代码后工作正常。
            \end{enumerate}

    \subsection{访存模块}
        \subsubsection{测试方法}
            \begin{enumerate}
            \item
            通过拨码开关输入访问存储(包括Ram、Flash、串口)的物理地址、数据、使能,%
            将读取到的数据通过LED显示出来,检查数据的正确性。
            \item
            使用状态机,在高频下连续地向存储中读写数据,检查数据的正确性。
            \end{enumerate}
            访问串口时需要在PC端同步接受、发送。
        \subsubsection{测试结果}
            Ram、Flash可在50MHz时钟下正常工作,%
            串口在25MHz以上时钟工作时,%
            连续两个时钟周期写入有一定几率出现错误,原因不详。%
            考虑到操作系统使用串口时不可能出现连续两个时钟访问串口,%
            此问题无需修复。

    \subsection{MMU模块}
        \subsubsection{功能概述}
            该模块实现虚拟地址到物理地址的转换,%
            并将物理地址转换为RAM、Flash、ROM的真实地址,%
            实现TLB表的查找、重填以及相关异常的触发。%

        \subsubsection{测试方法}
            主要测试地址转换过程与TLB表的查询过程。%
            由于拨码开关和LED等数量有限,%
            因此固定访存地址的低16位,%
            拨码开关只改变可能影响内存映射的高16位地址。%
            其余拨码开关通过组合,%
            查看不同的输出信号。%
            在VHDL代码中固化一个TLB表项,%
            通过拨码开关控制写入TLB。

        \subsubsection{测试用例}
            测试用例参见表\ref{mmu_module}
            \begin{table}[!hbp]
            \centering
            \caption{MMU模块测试用例}
            \label{mmu_module}
            \begin{tabularx}{\textwidth}{|l|X|}
            \hline
            类型 & 说明 \\
            \hline
            非映射地址转换 & 地址高16位选择非映射地址区间,%
                            观察输出的to\_physical\_addr信号是否正确。%
                            在非映射状态对RAM、Flash与串口均进行测试。   \\
            \hline
            映射地址转换 & 地址高16位选择映射地址区间,%
                            观察观察输出的to\_physical\_addr信号是否正确。%
                            对RAM、Flash与串口均进行测试。    \\
            \hline
            异常触发 & 不写入TLB表项进行映射地址的访存,触发TLBMiss异常。%
                        修改TLB表项,dirty位置1,触发TLBModify异常。%
                        修改输入的align\_type对齐方式,触发地址不对齐异常。    \\
            \hline
            串口状态位 & 串口状态位封装在MMU内部,%
                        在MMU模块测试串口是否可读可写。 \\
            \hline
            \end{tabularx}
            \end{table}

        \subsubsection{测试结果}
            该模块测试中发现了TLB命中策略的问题,%
            两个EntryLo页的命中与标准实现相反。%

            后在向勇老师的课件中,%
            找到一份TLB的详细实现方案,%
            在此基础上稍作修改即可实现TLB。

    \subsection{CP0模块}
        Assign to  : 孙皓

    \subsection{异常处理模块}
        Assign to  : 孙皓
