\section{指令片段测试}

    项目测试进入第三个阶段。%
    在前两阶段完成对CPU单指令运行的测试后,%
    此阶段完成对多条、不同类型指令组合的测试。%
    此阶段的测试目标为,%
    将不同指令组合所产生的问题解决。%
    同时,前两阶段的测试均使用手动时钟,%
    在此阶段换为实验板上提供的时钟,%
    对CPU在高频下的表现做测试。

    在此阶段,%
    我们也完成了我们调试工具的最初版本,%
    将可能需要查看的信号输出到debug模块,%
    通过拨码开关控制,可以查看不同的信号,%
    信号的值通过LED灯进行显示。

    \subsection{连续算数指令测试}
        \subsubsection{测试方法}
            对连续算数指令进行测试。%

            利用VHDL中的std\_logic\_vector生成一个32位数组,%
            以此数组作为ROM,%
            保存测试过程中所需的所有指令。%
            使用汇编语言写好测试指令,%
            在尾部加while死循环,%
            保证运算的结果可以保存在寄存器中。%
            用mips-linux-gnu-gcc进行编译,%
            将其中的二进制指令部分通过我们开发的romgen工具%
            转换为VHDL语言可以接受的形式。%
            至此测试指令准备完毕。

            按下rst开关后,将PC指向ROM起始地址,%
            即可开始运行指令片段。%

            运行结束后,通过拨码开关,%
            将结果寄存器中的数值显示在LED灯上。%
            通过调整拨码开关查看不同寄存器的值,%
            检查计算是否正确。

        \subsubsection{测试样例}
            测试样例参见表\ref{arithmetic}
            \begin{table}[!hbp]
            \centering
            \caption{算数指令测试样例}
            \label{arithmetic}
            \begin{tabularx}{\textwidth}{|l|X|}
            \hline
            类型 & 说明 \\
            \hline
            普通算数指令 & 包括add、sub、slt、shift等多类指令的组合,%
                        主要是对前一阶段单指令测试的补充,%
                        保证算数指令可以在组合的情况下工作。  \\
            \hline
            乘法指令 & 主要对乘法指令的速度是否满足需求进行测试,%
                        即在mult指令后紧跟mfhi、mflo指令,%
                        测试乘法模块是否正常  \\
            \hline
            综合算数测试 & 乘法指令与普通算数指令混合使用。 \\
            \hline
            \end{tabularx}
            \end{table}

        \subsubsection{问题与解决}
            该测试中未发现问题。%
            算数指令能够在高频下以各种组合正常执行。


    \subsection{连续访存指令测试}
        \subsubsection{测试方法}
            对连续访存指令进行测试。%

            测试指令仍保存在ROM中,%
            ROM的生成方式与算数指令相同。

            利用贾开的controller.py工具,%
            预先向Flash与RAM中写入一些数据。%
            按下rst运行,之后仍通过controller.py工具,%
            将RAM中的内容读出,与写入的内容做比较。%

        \subsubsection{测试样例}
            测试样例参见表\ref{memory}。
            \begin{table}[!hbp]
            \centering
            \caption{访存指令测试样例}
            \label{memory}
            \begin{tabularx}{\textwidth}{|l|X|}
            \hline
            类型 & 说明 \\
            \hline
            读Flash写RAM & 预先向Flash指定位置写入一段数据,%
                            测试指令中,通过while循环将该段数据读出,%
                            写入RAM的指定位置。%
                            之后通过controller.py读出RAM中的数据,%
                            与写入的数据进行对比。 \\
            \hline
            读RAM写RAM & 预先向RAM指定位置写入一段数据,%
                        测试指令中,通过while循环将该段数据读出,%
                        写入另一个RAM位置。%
                        之后通过controller.py读出RAM写入的数据,%
                        与写入的数据进行对比。 \\
            \hline
            LB与SB指令 & 预先向RAM指定位置写入一段数据,%
                        测试指令中,对于每一个4byte数据,%
                        用LB指令读取第一个byte的内容,%
                        之后连续三次SB指令写入后三个byte。%
                        通过controller.py读出该段RAM的数据,%
                        每连续四个byte都应该有相同的内容。 \\
            \hline
            串口连续输出 & 预先向RAM指定位置写入一段数据,%
                         测试指令中,通过一个while循环读出该段数据,%
                         分四次将每一个byte输出到串口地址。%
                         在电脑端将接收程序terminal的输出重定向到文件,%
                         与写入的文件进行对比。    \\
            \hline
            \end{tabularx}
            \end{table}

        \subsubsection{问题与解决}
            该测试中发现串口接收端存在问题。%
            第一次测试中发现电脑端乱码,%
            经过检查是波特率的设置出现了问题,%
            如果CPU端将时钟二分频,%
            则在电脑端需要将波特率变为二分之一。%
            第二次测试中发现电脑端接受的数据顺序相反,%
            经过检查是大小端的问题,%
            改变接受程序即可解决。

            通过该测试,证明了串口的输出在高频下是能够正常工作的。%
            为之后升级测试工具,将信号通过串口输出做好了准备。

    \subsection{模拟操作系统测试}
        \subsubsection{测试方法}
            模拟操作系统的工作过程进行测试。

            将一段测试代码用mips-linux-gnu-gcc进行编译,%
            之后修改编译出的二进制文件%
            (linux下可用bless工具),%
            将二进制文件前四个byte改为该段指令的条数。%
            之后将该二进制文件烧入Flash起始处。
            
            ROM中实现一段访操作系统bootloader的指令。%
            首先从Flash起始位置读入指令条数,%
            之后循环将指令读入RAM起始地址,%
            之后jump到RAM起始地址执行。

            由于前一测试已经将串口输出调通,%
            因此可以将代码的运行结果通过串口输出到电脑。

        \subsubsection{测试样例}
            Flash中的测试代码可以是任意mips32指令的组合。

        \subsubsection{问题与解决}
            该测试中未发现问题。%
            