\section{硬件部署}
    以CaCPU为例叙述如何将本工具应用于被测试模块。

    首先在被测试模块的实体加入两个输出信号:被查看信号和断点条件。

    \verbatiminput{diff/entity.diff}

    在结构体的并发处理语句中定义被查看信号和断点条件。%
    被查看信号的定义可使用工具自动生成,默认8对齐,未使用的用0填充。%
    断点条件未使用的部分必须用0填充。

    在本例中被查看信号共4096位,断点条件64位。%
    被查看的信号state类型为std\_logic\_vector,共3位,%
    用0填充5位以实现8对齐。%
    被查看的信号clk\_cpu类型为std\_logic,占1位,用0填充7位。%
    断点条件低32位为CPU当前PC,接着3位是state信号,剩余位用0填充。

    \verbatiminput{diff/arch.diff}
    
    在新的顶层模块中原件例化被调试的模块,并将输出端口对应起来。

    \verbatiminput{diff/cadb.diff}

    应将被调试模块的原时钟连接到bp\_debug模块的输入时钟,%
    将bp\_debug模块的输出时钟连接到被调试模块的时钟。

    为cadb分配串口管脚。
