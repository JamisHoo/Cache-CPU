\section{软件部署}
    首先编写配置文件,配置文件共三列。

    第一列是软件端查看变量值时的标签,接受不包含空白字符的字符串。

    第二列是VHDL源代码中实际被查看信号名称,%
    接收不包含空白字符的字符串,在被调试模块中应有同名信号,%
    且该信号不能为输出端口。

    第三列是被查看信号的范围。%
    对于std\_logic\_vector应标明范围,%
    对于std\_logic第三列空。

    配置文件支持\#风格的注释。

    运行软件端程序cadb gen <目标文件>,%
    信号定义部分VHDL代码将自动生成并保存在目标文件中。

    每次更新配置文件后都应该重新生成VHDL代码并将其更新到硬件中。

    运行cadb <串口设备路径>,调试工具将启动。

