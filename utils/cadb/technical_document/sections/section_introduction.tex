\section{引言}
    \subsection{开发信息}
        系统名称:硬件调试工具Cache Debugger

        开发者:
        \begin{minipage}[t]{0.8\linewidth}
        计23 李天润

        计23 胡津铭

        计23 孙皓
        \end{minipage}

    \subsection{内容简介}
        本文档介绍了硬件调试工具的功能设计以及软硬件部分实现细节,方便读者更深入了解本工具的运行方式,以及根据自己的需要对工具进行修改、二次开发。

        本文档假设读者已经阅读过调试工具的使用手册,并且了解硬件描述语言的基本应用以及元件例化的方法。

\section{设计原理}
    硬件逻辑分时序逻辑与组合逻辑,其中组合逻辑仅与输入有关,时序逻辑受时钟控制。

    在系内课程与FPGA有关的实验中,并没有涉及与延时相关的逻辑,比如振荡电路,因此FPGA芯片内部的硬件延时在设计硬件逻辑时一般不会被考虑,仅影响时钟频率的提高。%
    基于以上考虑,可以认为大多数实验中,硬件逻辑的执行是由时钟驱动的,因此控制了时钟也就控制了逻辑的执行。%
    而且,由于设计逻辑时不会将延时作为正面的逻辑功能的一部分,时钟沿无论延迟多久到达都不会产生负面作用,因此暂停时钟相当于将电路暂停,之后再次给入时钟信号即可无损地继续执行程序。

    本工具正是基于以上结论,结合调试的断点信息,产生被调试模块的时钟信号,使得被调试模块既可正常运行,又可以在给定的状态下暂停,查看调试信息。

    通过逻辑操作,本工具可以为被调试模块提供与调试工具的输入时钟信号相同频率的时钟信号,使得被调试模块在调试环境下有着与实际单独运行时几乎相同的表现。
