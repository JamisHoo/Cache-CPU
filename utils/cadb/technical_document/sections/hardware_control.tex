        \begin{tabularx}{\textwidth}{lll}
            \toprule
            端口名          & 端口方向  & 端口类型 \\
            \cmidrule(l){2-3}
            &
            \multicolumn{2}{X}{端口描述} \\
            \midrule
            clk             & in        & std\_logic \\
            \cmidrule(l){2-3}
            &
            \multicolumn{2}{X}{
                时钟输入。
            } \\
            \midrule
            clk\_o             & out        & std\_logic \\
            \cmidrule(l){2-3}
            &
            \multicolumn{2}{X}{
                由此模块产生的时钟信号,可直接作为被测试模块的时钟。
            } \\
            \midrule
            e             & in        & std\_logic \\
            \cmidrule(l){2-3}
            &
            \multicolumn{2}{X}{
                使能信号。
            } \\
            \midrule
            datas             & in        & std\_logic\_vector ( 4095 downto 0 ) \\
            \cmidrule(l){2-3}
            &
            \multicolumn{2}{X}{
                从被测试模块发来的信号值数据。
            } \\
            \midrule
            monitor             & in        & std\_logic\_vector ( 63 downto 0 ) \\
            \cmidrule(l){2-3}
            &
            \multicolumn{2}{X}{
                从被测试模块发来的监控信号数据,用于断点的监控。
            } \\
            \midrule
            serialport\_txd        & out      & std\_logic \\
            \cmidrule(l){2-3}
            &
            \multicolumn{2}{X}{
                串口写端口。
            } \\
            \midrule
            serialport\_rxd             & in        & std\_logic \\
            \cmidrule(l){2-3}
            &
            \multicolumn{2}{X}{
                串口读端口。
            } \\
            \bottomrule
        \end{tabularx}
