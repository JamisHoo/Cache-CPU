\section{高级应用}
    此部分涉及较多细节,请在理解本文档前述部分之后再进行阅读。
    \subsection{断点调试}
        本调试工具的断点调试可以按照64位均匹配的方式执行,也可以根据需要进行一定的通配匹配。

        对于包含通配符的断点信息,来自客户端的指令的第二个参数不再是0xFFFFFFFF,而在需要通配的位上为0。%
        断点的匹配是将断点值与监控值作异或,如果结果为0x00000000,则表示到达断点。%
        考虑通配符后,异或的结果会再与通配值进行与操作,将被设为通配的位置0,从而使其不再参与断点匹配。

    \subsection{观察信号变长}
        默认文件中,观察信号长度为4096bit,使用中可以将CPU顶层模块的所有信号以及通用寄存器、CP0寄存器的所有信号发到客户端,
        而且发送这一长度的观察信号消耗的时间在通过终端查看的情况下可以忽略。

        使用中,考虑到不同的使用者可能有不同的需要,比如自动调试中希望每次发送更少的信号值或者需要发送更多的信号,在此给出改变观察信号长度的方法。

        % 不必须
        % 需要注意的是,数据发送以字节为单位,因此需要保证观察信号的长度为8的倍数。

        改变观察信号长度,需要将被调试模块中的输出数据长度改变,调试模块的输入数据长度改变。

        接下来,需要将数据发送部分的字节编号范围改为需要的数值,此处需修改com\_debug.vhd中的常量slc\_max。

        最后,需要在软件端进行相应修改,使其一次性接受的信号数量匹配。
