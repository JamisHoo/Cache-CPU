\section{引言}
    \subsection{编写的目}
        计原32位MIPS实验是在计原16位大实验的基础上的扩展。
        在实验原理方面与多门课程相结合,涉及到操作系统、软件工程、编译原理等等多个方面,
        实验初期学习曲线较陡,在原理方面比较难以掌控。
        在编程实现方面涉及到大规模VHDL代码的书写,需要对数字逻辑设计有比较清晰的思路,工作量非常大。

        因此,为控制整个开发过程,明确项目需求与目标,编写此需求文档。 

        文档预期读者为任务提出者:刘卫东老师、白晓颖老师。
    \subsection{背景}
        系统名称:32位MIPS处理器

        任务提出者:
        $
        \begin{minipage}[t]{0.9\linewidth}
        计原课:刘卫东老师

        软工课:白晓颖老师
        \end{minipage}
        $

        开发者:
        $
        \begin{minipage}[t]{0.9\linewidth}
        计23 李天润

        计23 胡津铭

        计23 孙皓
        \end{minipage}
        $

    \subsection{定义}
        以下是此次32位MIPS实验中需要用到的专业名词定义
        \begin{longtable}{|p{0.3\linewidth}|p{0.5\linewidth}|}
        \caption{定义列表}
        \hline
        名词 & 描述 \\
        \hline
        MIPS & 1个 \\
        \hline
        多周期CPU & 1个 \\
        \hline
        CP0 & 1个 \\
        \hline
        MMU & 1个 \\
        \hline
        TLB & 1个 \\
        \hline
        虚拟地址 & 1个 \\
        \hline
        ALU & 1个 \\
        \hline
        数据通路 & Xlinx ISE 14.7 \\
        \hline
        异常 & Xlinx ISE 14.7 \\
        \hline
        中断 & Xlinx ISE 14.7 \\
        \hline
        \end{longtable}

    \subsection{参考资料}
        实验指导文档

        OsLab实验参考文档

        贾开Report

        《计算机组成和设计硬件/软件接口》

        《See MIPS Run》


