\documentclass[12pt,a4paper]{article}
\usepackage{CJKutf8}
\usepackage{indentfirst}
\setlength{\parindent}{2em}
\usepackage{xcolor}
\usepackage{geometry}
\geometry{left=2.5cm,right=2.5cm,top=2.0cm,bottom=2.5cm}
\usepackage{mdwlist}
\usepackage{array}
\usepackage{caption}
\usepackage{longtable}
\usepackage{supertabular}
\usepackage{amsmath}
\usepackage{graphicx}

\begin{document}
\begin{CJK}{UTF8}{gkai}
\author{Cache小组}
\title{32位MIPS处理器实验需求文档}
\maketitle
\end{CJK}

\begin{CJK}{UTF8}{gbsn}
\tableofcontents

\section{引言}
	\subsection{编写的目}
		为控制整个开发过程,明确项目需求与目标,编写此需求文档。\par
		文档预期读者为任务提出者:刘卫东老师、白晓颖老师
	\subsection{背景}
		\quad 系统名称:32位MIPS处理器\par
		\quad 任务提出者:计原课:刘卫东老师\par
		\quad 软工课:白晓颖老师\par
		\quad 开发者:计23 李天润\par
		\quad\quad		计23 胡津铭\par
		\quad\quad		计23 孙皓\par
	\subsection{定义}
		暂无
	\subsection{参考资料}
		实验指导文档\par
		oslab实验参考文档\par
		贾开report\par
		《计算机组成和设计硬件/软件接口》\par
		《See MIPS Run》\par

\section{功能需求}
	\subsection{CPU}
		\subsubsection{ALU}
			\begin{description*}
				\setlength{\itemsep}{1pt}
				\item[\quad 功能需求:]\hfill\par
					1、完成数据和地址的算术、逻辑和移位运算,两个输入数据根据ALUOp得出输出结果。\par
					2、根据指令系统的需求,完成指令系统中乘法之外的算术指令。\par
				%\vspace{2ex}		%use if you want to expand 
				\item[\quad 实现方式:]\hfill\par
					1、ALU两个输入数据,ALUOp目前整理出9种运算。\par
					2、输出最终运算结果,标志位三种。\par
			\end{description*}

		\subsubsection{乘法器}
			\begin{description*}
				\setlength{\itemsep}{1pt}
				\item[\quad 功能需求:]\hfill\par
					1、完成乘法功能,结果保存在LO和HI中,可以访问计算结果
				%\vspace{2ex}		%use if you want to expand 
				\item[\quad 实现方式:]\hfill\par
					1、使用IPCore实现\par
					2、是否与ALU放在一起,之后测试乘法器的速度再做决定。\par
					3、MFLO,MFHI,MTLO,MTHI,MULT指令查看。
			\end{description*}

		\subsubsection{CP0}
			\begin{description*}
				\setlength{\itemsep}{1pt}
				\item[\quad 功能需求:]\hfill\par
					1、辅助操作系统,实现内存管理、异常处理等方面的硬件支持。\par
					2、对硬件运行过程进行控制
				%\vspace{2ex}		%use if you want to expand 
				\item[\quad 实现方式:]\hfill\par
					1、MMU、TLB相关的寄存器:Index、EntryLo0、EntryLo1、EntryHi\par
					2、异常处理相关寄存器:BadVAddr、Status、Cause、EPC、EBase\par
					3、时钟相关的寄存器:Count、Compare\par
					4、具体实现方式参照下面两个小节
			\end{description*}
			
		\subsubsection{MMU与TLB}
			\begin{description*}
				\setlength{\itemsep}{1pt}
				\item[\quad 功能需求:]\hfill\par
					1、实现虚拟地址(线性地址)到物理地址的转换\par
					2、实现相对应的异常处理:TLBS、TLBL、TLB Modified
				%\vspace{2ex}		%use if you want to expand 
				\item[\quad 硬件支持:]\hfill\par
					1、TLB表项并行查询EntryHi部分。\par
					2、如果查找到,EntryLo部分直接与低12位结合得到真实的物理地址。\par
					3、如果未查找到,触发TLBMiss异常:设置Cause寄存器中的ExcCode为TLB异常,EPC寄存器为当前指令的地址,BadvAddr寄存器为错误的地址,之后触发一个异常,由操作系统接管。
				\item[\quad 操作系统:]\hfill\par
					1、进入异常处理向量(trap/vector.S)\par
					2、跳转到处理函数部分(trap/exception.S)\par
					3、先保存异常现场,之后进入mips-trap函数(trap/trap.c)\par
					4、根据cause寄存器的值进行分类,调用handle-tlbmiss函数\par
					5、得到异常地址对应的物理页号,之后进行tlb-refill(include/thumips-tlb.h)\par
					6、操作系统维护index,写EntryLo0、EntryLo1、EntryHi、 PageMask、 Index五个寄存器,之后用tlbwi写到TLB的Index项\par
					7、异常返回,重新执行取地址命令
			\end{description*}

		\subsubsection{中断处理}
			\begin{description*}
				\setlength{\itemsep}{1pt}
				\item[\quad 功能需求:]\hfill\par
					1、实现对异常和外部中断的处理。\par
					2、实现的中断和异常有外部中断、地址不对齐异常、系统调用、未定义的指令异常\par
					3、外部中断:键盘、通讯端口
				%\vspace{2ex}		%use if you want to expand 
				\item[\quad 硬件支持:]\hfill\par
					1、异常处理,在数据通路上添加异常处理部分,在多周期的任意一个周期检测到异常后,设置Cause、EPC寄存器,之后跳转到异常处理向量基地址,之后操作系统接管。\par
					2、外部中断处理,设置外部中断检查信号,外部设备触发时,异步地将检查信号置1;每个指令IF检查该信号,若为1则进入外部中断处理,否则正常执行指令。\par
					3、时钟中断处理
				\item[\quad 操作系统:]\hfill\par
					1、进入异常处理向量(trap/vector.S)\par
					2、跳转到处理函数部分(trap/exception.S)\par
					3、先保存异常现场,之后进入mips-trap函数(trap/trap.c)\par
					4、根据cause寄存器的值进行分类,调用interrupt-handler 、syscall等等函数进行处理,如果是地址不对齐异常,未定义的指令或未定义的异常,直接退出。\par
					5、异常返回,重新执行取地址命令
			\end{description*}

		\subsubsection{其他功能部件}
			\paragraph{PC}
				\begin{description*}
					\setlength{\itemsep}{2pt}
					\item[\quad 功能需求:]\hfill\par
						1、实现PC的多种变换方式:正常执行、分支、跳转\par
						2、实现异常处理时对PC的处理,跳转到异常处理向量部分
						%\vspace{2ex}		%use if you want to expand 
					\item[\quad 实现方式:]\hfill\par
						1、将PC与ALU结合实现PC的改变,同时通过多路选择器选择PC的变化方式
				\end{description*}

			\paragraph{寄存器堆}
				\begin{description*}
					\setlength{\itemsep}{2pt}
					\item[\quad 功能需求:]\hfill\par
						1、实现通用寄存器,以及在数据通路中的读写控制
					%\vspace{2ex}		%use if you want to expand 
					\item[\quad 实现方式:]\hfill\par
						1、在CPU中实现寄存器堆并实现读写控制
				\end{description*}
			
			\paragraph{串口}
				\begin{description*}
					\setlength{\itemsep}{2pt}
					\item[\quad 功能需求:]\hfill\par
						1、可能要与计算机交互,如果需要,会添加串口需求
					%\vspace{2ex}		%use if you want to expand 
					\item[\quad 实现方式:]\hfill\par
						1、开发过程中,根据实际需求进行串口开发
				\end{description*}

		\subsubsection{扩展部分}
			目前没有关于扩展部分的想法,先完成基本部分。\par
			网络部分比较有明确的开发目标,但组内无操作系统开发经验\par
			测试工具部分没有明确的开发目标,不太清楚如何入手开发

		\subsubsection{指令集与数据通路}
			实现多周期CPU,针对指令集设计数据通路。\par
			仿照《软件硬件接口》书中的多周期CPU进行设计。\par
			得到指令之后对指令进行解码,得到相应的控制线。在多周期的不同周期利用控制线对数据通路进行控制。

	\subsection{BIOS}
		\begin{description*}
			\setlength{\itemsep}{1pt}
			\item[\quad 准备阶段:]\hfill\par
				1、操作系统和其他程序烧写到Flash中\par
				2、bootasm.S编译出的二进制文件写到ROM中\par
				3、CPU写到FPGA中。
			%\vspace{2ex}		%use if you want to expand 
			\item[\quad 启动阶段:]\hfill\par
				1、从ROM中的bootasm.S启动\par
				2、将Flash中的操作系统拷贝到RAM中\par
				3、然后控制权交给操作系统。
		\end{description*}

\section{性能需求}
	实现多周期CPU保证性能
\section{运行环境需求}
	\subsection{设备}
		硬件部采用老师提供的开发板。\par
		FPGA: Xlinx Spartan6 xc6slx100\par
		CPLD: XC95144\par
		RAM: 两片 61lv102416 32位RAM \par
		程序通过Xlinx ISE开发套件进行开发。

	\subsection{控制}
		控制部分采用PS/2键盘作为输入设备,FPGA进行计算与处理之后通过VGA进行输出。如果有需要使用串口与PC进行通信,扩展功能可以采用网口实现开发板与PC之间的通信
\newpage
\end{CJK}
\end{document}
