\documentclass{article}
\usepackage[usenames,dvipsnames]{color} % Required for custom colors
\usepackage{graphicx} % Required to insert images
\usepackage{listings} % Required for insertion of code
\usepackage{amsmath} % Required for some math formulas
\usepackage{enumitem} % Required for customed enum
\usepackage{url} % Auto line break in URL
\usepackage{tabularx} % Auto line break in tabular


\def\UrlBreaks{\do\[\do\\\do\]\do\^\do\_\do\`\do\.\do\@\do\\\do\/\do\!\do\_\do\|\do\;\do\>\do\]\do\)\do\,\do\?\do\'\do+\do\=\do\#}

\usepackage{hyperref} % add clickable link to contents
\hypersetup {
    colorlinks = false, % links colored or not
    hidelinks = true, % border
    linkcolor = blue, % color TOC links in blue
    urlcolor = red, % color URLs in red
    linktoc = all % 'all' will create links for everything in the TOC
}

% Uncomment the lines below if you wish to use Chinese characters
\usepackage{fontspec}  % Chinese characters support
\usepackage{indentfirst} % Add indent at the first paragraph
\XeTeXlinebreaklocale "zh" % Automatic line break
\XeTeXlinebreakskip = 0pt plus 1pt % Automatic line break
% \setmainfont{STSong} % Use a Chinese font
% \setmainfont[Path = ./]{AdobeSongStd-Light.otf} % use font in ./
\setmainfont[Path = ./fonts/]{AdobeFangsongStd-Regular.otf}

% Use Chinese for figure name
\renewcommand\figurename{图}

% Use Chinese for contents name
\renewcommand\contentsname{目录}

% Use Chinese for table name
\renewcommand\tablename{表}

% contents depth
\setcounter{tocdepth}{4}

% numbered paragraph
\setcounter{secnumdepth}{4}

%----------------------------------------------------------------------------------------
%	TITLE SECTION
%----------------------------------------------------------------------------------------

\newcommand{\horrule}[1]{\rule{\linewidth}{#1}} % Create horizontal rule command with 1 argument of height

\title{	
\normalfont \normalsize 
% \textsc{university, school or department name} \\ [25pt] % Your university, school and/or department name(s)
\horrule{0.5pt} \\[0.4cm] % Thin top horizontal rule
\huge 32位MIPS处理器实验需求文档 \\ % The assignment title
\horrule{2pt} \\[0.5cm] % Thick bottom horizontal rule
}

\author{Cache小组} % Your name

\date{\normalsize\the\year 年\the\month 月\the\day 日} % Today's date or a custom date




% document begin
\begin{document}
\maketitle % Print the title

\tableofcontents % Print the contents

\newpage % blank after contents

\section{引言}
    \subsection{编写目的}
        测试在硬件中的作用、为后人作参考等等

        Assign to : 李天润

    \subsection{测试方案}
        本项目的测试工作从硬件模块的单元测试开始,%
        逐渐将各个模块组合联调,%
        使其最终能够正常运行一个操作系统。%
        总体测试框架如下:
        \begin{enumerate}
        \item
            单元测试
        \item
            单指令测试
        \item
            指令片段测试
        \item
            系统测试
        \end{enumerate}
\newpage
\section{功能需求}
    \subsection{CPU}
        \subsubsection{ALU}
            功能需求:
            \begin{enumerate}
            \item
            完成数据和地址的算术、逻辑和移位运算,两个输入数据根据ALUOp得出输出结果。
            \item
            根据指令系统的需求,完成指令系统中乘法之外的算术指令。
            \end{enumerate}

            实现方式:
            \begin{enumerate}
            \item
            ALU两个输入数据,ALUOp目前整理出9种运算。
            \item
            输出最终运算结果,标志位三种。
            \end{enumerate}

        \subsubsection{乘法器}
            功能需求:
            \begin{enumerate}
            \item
            完成乘法功能,结果保存在LO和HI中,可以访问计算结果。
            \end{enumerate}

            实现方式:
            \begin{enumerate}
            \item
            使用IPCore实现。
            \item
            是否与ALU放在一起,之后测试乘法器的速度再做决定。
            \item
            MFLO,MFHI,MTLO,MTHI,MULT指令查看。
            \end{enumerate}

        \subsubsection{CP0}
            功能需求:
            \begin{enumerate}
            \item
            辅助操作系统,实现内存管理、异常处理等方面的硬件支持。
            \item
            对硬件运行过程进行控制。
            \end{enumerate}

            实现方式:
            \begin{enumerate}
            \item
            MMU、TLB相关的寄存器:Index、EntryLo0、EntryLo1、EntryHi。
            \item
            异常处理相关寄存器:BadVAddr、Status、Cause、EPC、EBase。
            \item
            时钟相关的寄存器:Count、Compare。
            \item
            具体实现方式参照下面两个小节。
            \end{enumerate}

        \subsubsection{MMU与TLB}
            功能需求:
            \begin{enumerate}
            \item
            实现虚拟地址(线性地址)到物理地址的转换。
            \item
            实现相对应的异常处理:TLBS、TLBL、TLB Modified。
            \end{enumerate}

            实现方式:
            \begin{enumerate}
            \item
                硬件支持:
                \begin{enumerate}
                \item
                TLB表项并行查询EntryHi部分。
                \item
                如果查找到,EntryLo部分直接与低12位结合得到真实的物理地址。
                \item
                如果未查找到,触发TLBMiss异常:%
                设置Cause寄存器中的ExcCode为TLB异常,%
                EPC寄存器为当前指令的地址,%
                BadvAddr寄存器为错误的地址,%
                之后触发一个异常,由操作系统接管。
                \end{enumerate}
            \item
                操作系统:
                \begin{enumerate}
                \item
                进入异常处理向量(trap/vector.S)。
                \item
                跳转到处理函数部分(trap/exception.S)。
                \item
                先保存异常现场,之后进入mips\_trap函数(trap/trap.c)。
                \item
                根据cause寄存器的值进行分类,调用handle\_tlbmiss函数。
                \item
                得到异常地址对应的物理页号,%
                之后进行tlb\_refill(include/thumips\_tlb.h)。
                \item
                操作系统维护index,%
                写EntryLo0、EntryLo1、EntryHi、 PageMask、Index五个寄存器,%
                之后用tlbwi写到TLB的Index项。
                \item
                异常返回,重新执行取地址命。%
                \end{enumerate}
            \end{enumerate}
        \subsubsection{中断处理}
            功能需求:
            \begin{enumerate}
            \item
            实现对异常和外部中断的处理。
            \item
            实现的中断和异常有外部中断、地址不对齐异常、%
            系统调用、未定义的指令异常。
            \item
            外部中断:键盘、通讯端口。
            \end{enumerate}

            实现方式:
            \begin{enumerate}
            \item
            硬件支持
                \begin{enumerate}
                \item
                异常处理,在数据通路上添加异常处理部分,%
                在多周期的任意一个周期检测到异常后,%
                设置Cause、EPC寄存器,之后跳转到异常处理向量基地址,%
                之后操作系统接管。
                \item
                外部中断处理,设置外部中断检查信号,外部设备触发时,%
                异步地将检查信号置1;每个指令IF检查该信号,%
                若为1则进入外部中断处理,否则正常执行指令。
                \item
                时钟中断处理。
                \end{enumerate}
            \item
            操作系统
                \begin{enumerate}
                \item
                进入异常处理向量(trap/vector.S)。
                \item
                跳转到处理函数部分(trap/exception.S)。
                \item
                先保存异常现场,之后进入mips\_trap函数(trap/trap.c)。
                \item
                根据cause寄存器的值进行分类,%
                调用interrupt\_handler、syscall等等函数进行处理,%
                如果是地址不对齐异常,未定义的指令或未定义的异常,直接退出。
                \item
                异常返回,重新执行取地址命令
                \end{enumerate}
            \end{enumerate}
        \subsubsection{其他功能部件}
            \paragraph{PC}
                \mbox{} \\ 

                功能需求:
                \begin{enumerate}
                \item
                实现PC的多种变换方式:正常执行、分支、跳转。
                \item
                实现异常处理时对PC的处理,跳转到异常处理向量部分。
                \end{enumerate}

                实现方式
                \begin{enumerate}
                \item
                将PC与ALU结合实现PC的改变,同时通过多路选择器选择PC的变化方式。
                \end{enumerate}

            \paragraph{寄存器堆}
                \mbox{} \\ 

                功能需求:
                \begin{enumerate}
                \item
                实现通用寄存器,以及在数据通路中的读写控制。
                \end{enumerate}

                实现方式
                \begin{enumerate}
                \item
                在CPU中实现寄存器堆并实现读写控制。
                \end{enumerate}

            \paragraph{串口}
                \mbox{} \\

                功能需求:
                \begin{enumerate}
                \item
                可能要与计算机交互,如果需要,会添加串口需求。
                \end{enumerate}

                实现方式
                \begin{enumerate}
                \item
                开发过程中,根据实际需求进行串口开发。
                \end{enumerate}
        \subsubsection{扩展部分}
            目前没有关于扩展部分的想法,先完成基本部分。

            网络部分比较有明确的开发目标,但组内无操作系统开发经验。

            测试工具部分没有明确的开发目标,不太清楚如何入手开发。

        \subsubsection{指令集与数据通路}
            实现多周期CPU,针对指令集设计数据通路。

            仿照《软件硬件接口》书中的多周期CPU进行设计。

            得到指令之后对指令进行解码,得到相应的控制线。%
            在多周期的不同周期利用控制线对数据通路进行控制。
    
    \subsection{BIOS}
        准备阶段:操作系统和其他程序烧写到Flash中,%
        bootasm.S编译出的二进制文件写到ROM中,再把CPU写到FPGA中。

        启动阶段:从ROM中的bootasm.S启动,%
        将Flash中的操作系统拷贝到RAM中,然后控制权交给操作系统。



\newpage
\section{性能需求}
    实现多周期CPU保证性能。



\newpage
\section{运行环境需求}
    \subsection{设备}
        \begin{table}[!hbp]
        \centering
        \caption{需要实现的CP0寄存器}
        \begin{tabular}{|l|l|}
        \hline
        环境 & 描述 \\
        \hline
        FPGA & Xilinx Spartan6 xc6slx100 \\
        \hline
        RAM & 32-bit字长, 4块, 共8MB \\
        \hline
        Flash & 16-bit 字长, 共8MB \\
        \hline
        CPLD & 与FPGA相连, 用于I/O \\
        \hline
        串口 & 2个 \\
        \hline
        USB 串口 & 1个 \\
        \hline
        ps/2 接口 & 1个 \\
        \hline
        以太网接口 & 1个 \\
        \hline
        VGA 接口 & 1个 \\
        \hline
        以太网接口 & 1个 \\
        \hline
        VHDL开发环境 & Xlinx ISE 14.7 \\
        \hline
        \end{tabular}
        \end{table}

    \subsection{控制}
        控制部分采用PS/2键盘作为输入设备,%
        FPGA进行计算与处理之后通过VGA进行输出。%
        如果有需要使用串口与PC进行通信,%
        扩展功能可以采用网口实现开发板与PC之间的通信。



\newpage
\section{附录}
    \subsection{指令系统}
    \begin{figure}[!hbp]
            \centering
            \includegraphics[width=\textwidth]{chart/insert1.jpg}
    \end{figure}
    
    \begin{figure}[!hbp]
            \centering
            \includegraphics[width=\textwidth]{chart/insert2.jpg}
    \end{figure}
    
    \begin{figure}[!hbp]
            \centering
            \includegraphics[width=\textwidth]{chart/insert3.jpg}
    \end{figure}

    \begin{figure}[!hbp]
            \centering
            \includegraphics[width=\textwidth]{chart/insert4.jpg}
    \end{figure}

    \begin{figure}[!hbp]
            \centering
            \includegraphics[width=\textwidth]{chart/insert5.jpg}
    \end{figure}

    \begin{figure}[!hbp]
            \centering
            \includegraphics[width=\textwidth]{chart/insert6.jpg}
    \end{figure}

    \begin{figure}[!hbp]
            \centering
            \includegraphics[width=\textwidth]{chart/insert7.jpg}
    \end{figure}
\newpage

\end{document}
