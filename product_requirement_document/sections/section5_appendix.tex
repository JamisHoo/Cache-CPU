\section{附录}
    \subsection{指令系统}
    指令系统参考了MIPS Instructions Reference。%
    其中Encoding部分标记为-的表示可以为任意01组合,%
    但实际编程实现中应当做0处理。

    \begin{table}[!hbp]
    \begin{tabularx}{\textwidth}{|c|X|}
        \hline
        Syntax: & addiu \$d, \$s, \$t \\
        \hline
        Description: & Adds a register and a sign-extended immediate value and stores the result in a register \\
        \hline
        Operation: & \$t=\$s + imm; \\
        \hline
        Encoding: & 0010 00ss ssst tttt iiii iiii iiii iiii \\
        \hline
    \end{tabularx}
    \end{table}

    \begin{table}[!hbp]
    \begin{tabularx}{\textwidth}{|c|X|}
        \hline
        Syntax: & addu \$d, \$s, \$t \\
        \hline
        Description: & Adds two registers and stores the result in a register \\
        \hline
        Operation: & \$d = \$s + \$t; \\
        \hline
        Encoding: & 0000 00ss ssst tttt dddd d000 0010 0001 \\
        \hline
    \end{tabularx}
    \end{table}

    
    \begin{table}[!hbp]
    \begin{tabularx}{\textwidth}{|c|X|}
        \hline
        Syntax: & slt \$d, \$s, \$t \\
        \hline
        Description: & If \$s is less than \$t, \$d is set to one. It gets zero otherwise. \\
        \hline
        Operation: & if \$s < \$t \$d = 1 else \$d = 0; \\
        \hline
        Encoding: & 0000 00ss ssst tttt dddd d000 0010 1010 \\
        \hline
    \end{tabularx}
    \end{table}

    \begin{table}[!hbp]
    \begin{tabularx}{\textwidth}{|c|X|}
        \hline
        Syntax: & slti \$t, \$s, imm \\
        \hline
        Description: & If \$s is less than immediate, \$t is set to one. It gets zero otherwise. \\
        \hline
        Operation: & if \$s < imm \$t = 1 else \$t = 0 \\
        \hline
        Encoding: & 0010 10ss ssst tttt iiii iiii iiii iiii \\
        \hline
    \end{tabularx}
    \end{table}

    \begin{table}[!hbp]
    \begin{tabularx}{\textwidth}{|c|X|}
        \hline
        Syntax: & sltiu \$t, \$s, imm \\
        \hline
        Description: & If \$s is less than the unsigned immediate, \$t is set to one. It gets zero otherwise. \\
        \hline
        Operation: & if \$s < imm \$t = 1 else \$t = 0; \\
        \hline
        Encoding: & 0010 11ss ssst tttt iiii iiii iiii iiii \\
        \hline
    \end{tabularx}
    \end{table}
    
    \begin{table}[!hbp]
    \begin{tabularx}{\textwidth}{|c|X|}
        \hline
        Syntax: & sltu \$d, \$s, \$t \\
        \hline
        Description: & If \$s is less than \$t, \$d is set to one. It gets zero otherwise. \\
        \hline
        Operation: & if \$s < \$t \$d = 1 else \$d = 0; \\
        \hline
        Encoding: & 0000 00ss ssst tttt dddd d000 0010 1011 \\
        \hline
    \end{tabularx}
    \end{table}

    \begin{table}[!hbp]
    \begin{tabularx}{\textwidth}{|c|X|}
        \hline
        Syntax: & subu \$d, \$s, \$t \\
        \hline
        Description: & Subtracts two registers and stores the result in a register \\
        \hline
        Operation: & \$d = \$s - \$t; \\
        \hline
        Encoding: & 0000 00ss ssst tttt dddd d000 0010 0011 \\
        \hline
    \end{tabularx}
    \end{table}

    \begin{table}[!hbp]
    \begin{tabularx}{\textwidth}{|c|X|}
        \hline
        Syntax: & mult \$s, \$t \\
        \hline
        Description: & Multiplies \$s by \$t and stores the result in \$HI|\$LO. \\
        \hline
        Operation: & \$HI|\$LO = \$s * \$t; \\
        \hline
        Encoding: & 0000 00ss ssst tttt 0000 0000 0001 1000 \\
        \hline
    \end{tabularx}
    \end{table}

    \begin{table}[!hbp]
    \begin{tabularx}{\textwidth}{|c|X|}
        \hline
        Syntax: & mflo \$d \\
        \hline
        Description: & The contents of register LO are moved to the specified register. \\
        \hline
        Operation: & \$d = \$LO; \\
        \hline
        Encoding: & 0000 0000 0000 0000 dddd d000 0001 0010 \\
        \hline
    \end{tabularx}
    \end{table}

    \begin{table}[!hbp]
    \begin{tabularx}{\textwidth}{|c|X|}
        \hline
        Syntax: & mfhi \$d \\
        \hline
        Description: & The contents of register HI are moved to the specified register. \\
        \hline
        Operation: & \$d = \$HI; \\
        \hline
        Encoding: & 0000 0000 0000 0000 dddd d000 0001 0000 \\
        \hline
    \end{tabularx}
    \end{table}

    \begin{table}[!hbp]
    \begin{tabularx}{\textwidth}{|c|X|}
        \hline
        Syntax: & mtlo \$s \\
        \hline
        Description: & The contents of register LO are replaced by the specified register. \\
        \hline
        Operation: & \$LO = \$s; \\
        \hline
        Encoding: & 0000 00ss sss0 0000 0000 0000 0001 0011 \\
        \hline
    \end{tabularx}
    \end{table}

    \begin{table}[!hbp]
    \begin{tabularx}{\textwidth}{|c|X|}
        \hline
        Syntax: & mthi \$s \\
        \hline
        Description: & The contents of register HI are replaced by the specified register. \\
        \hline
        Operation: & \$HI = \$s; \\
        \hline
        Encoding: & 0000 00ss sss0 0000 0000 0000 0001 0001 \\
        \hline
    \end{tabularx}
    \end{table}

    \begin{table}[!hbp]
    \begin{tabularx}{\textwidth}{|c|X|}
        \hline
        Syntax: & and \$d, \$s, \$t \\
        \hline
        Description: & Bitwise ands two registers and stores the result in a register. \\
        \hline
        Operation: & \$d = \$s \& \$t; \\
        \hline
        Encoding: & 0000 00ss ssst tttt dddd d000 0010 0100 \\
        \hline
    \end{tabularx}
    \end{table}

    \begin{table}[!hbp]
    \begin{tabularx}{\textwidth}{|c|X|}
        \hline
        Syntax: & andi \$t, \$s, imm \\
        \hline
        Description: & Bitwise ands a register and an immediate value and stores the result in a register. \\
        \hline
        Operation: & \$t = \$s \& imm; \\
        \hline
        Encoding: & 0011 00ss ssst tttt iiii iiii iiii iiii \\
        \hline
    \end{tabularx}
    \end{table}

\clearpage  %the maximize number of picutres/tables without this command is 18
            %for more information, search for 'LaTeX Error: Too many unprocessed floats'

    \begin{table}[!hbp]
    \begin{tabularx}{\textwidth}{|c|X|}
        \hline
        Syntax: & lui \$t, imm \\
        \hline
        Description: & The immediate value is shifted left 16 bits and stored in the register.
                         The lower 16 bits are zeroes. \\
        \hline
        Operation: & \$t = (imm << 16); \\
        \hline
        Encoding: & 0011 11-- ---t tttt iiii iiii iiii iiii \\
        \hline
    \end{tabularx}
    \end{table}

    \begin{table}[!hbp]
    \begin{tabularx}{\textwidth}{|c|X|}
        \hline
        Syntax: & nor \$d, \$s, \$t \\
        \hline
        Description: & Bitwise logical ors two registers and stores the logical not result in a register. \\
        \hline
        Operation: & \$d = $\sim$ (\$s | \$t); \\
        \hline
        Encoding: & 0000 00ss ssst tttt dddd d000 0010 0111 \\
        \hline
    \end{tabularx}
    \end{table}

    \begin{table}[!hbp]
    \begin{tabularx}{\textwidth}{|c|X|}
        \hline
        Syntax: & or \$d, \$s, \$t \\
        \hline
        Description: & Bitwise logical ors two registers and stores the result in a register. \\
        \hline
        Operation: & \$d = \$s | \$t; \\
        \hline
        Encoding: & 0000 00ss ssst tttt dddd d000 0010 0101 \\
        \hline
    \end{tabularx}
    \end{table}

    \begin{table}[!hbp]
    \begin{tabularx}{\textwidth}{|c|X|}
        \hline
        Syntax: & ori \$t, \$s, imm \\
        \hline
        Description: & Bitwise ors a register and an immediate value and stores the result in a register. \\
        \hline
        Operation: & \$t = \$s | imm; \\
        \hline
        Encoding: & 0011 01ss ssst tttt iiii iiii iiii iiii \\
        \hline
    \end{tabularx}
    \end{table}

    \begin{table}[!hbp]
    \begin{tabularx}{\textwidth}{|c|X|}
        \hline
        Syntax: & xor \$d, \$s, \$t \\
        \hline
        Description: & Exclusive ors two registers and stores the result in a register. \\
        \hline
        Operation: & \$d = \$s $\oplus$ \$t; \\
        \hline
        Encoding: & 0000 00ss ssst tttt dddd d--- --10 0110 \\
        \hline
    \end{tabularx}
    \end{table}

    \begin{table}[!hbp]
    \begin{tabularx}{\textwidth}{|c|X|}
        \hline
        Syntax: & xori \$t, \$s, imm \\
        \hline
        Description: & Bitwise exclusive ors a register and an immediate
                         value and stores the result in a register. \\
        \hline
        Operation: & \$t = \$s $\oplus$ imm; \\
        \hline
        Encoding: & 0011 10ss ssst tttt iiii iiii iiii iiii \\
        \hline
    \end{tabularx}
    \end{table}

    \begin{table}[!hbp]
    \begin{tabularx}{\textwidth}{|c|X|}
        \hline
        Syntax: & sll \$d, \$t, h \\
        \hline
        Description: & Shifts a register value left by the shift amount listed in 
                        the instruction and places the result in a third register.
                         Zeroes are shifted in. \\
        \hline
        Operation: & \$d = \$t << h \\
        \hline
        Encoding: & 0000 00ss ssst tttt dddd dhhh hh00 0000 \\
        \hline
    \end{tabularx}
    \end{table}

    \begin{table}[!hbp]
    \begin{tabularx}{\textwidth}{|c|X|}
        \hline
        Syntax: & sllv \$d, \$t, \$s \\
        \hline
        Description: & Shifts a register value left by the value 
                        in a second register and places the result 
                        in a third register. Zeroes are shifted in. \\
        \hline
        Operation: & \$d = \$t << \$s; \\
        \hline
        Encoding: & 0000 00ss ssst tttt dddd d--- --00 0100 \\
        \hline
    \end{tabularx}
    \end{table}

    \begin{table}[!hbp]
    \begin{tabularx}{\textwidth}{|c|X|}
        \hline
        Syntax: & sra \$d, \$t, h \\
        \hline
        Description: & Shifts a register value right by the shift 
                        amount (shamt) and places the value in the 
                        destination register. The sign bit is shifted in. \\
        \hline
        Operation: & \$d = \$t >> h; \\
        \hline
        Encoding: & 0000 00-- ---t tttt dddd dhhh hh00 0011 \\
        \hline
    \end{tabularx}
    \end{table}

    \begin{table}[!hbp]
    \begin{tabularx}{\textwidth}{|c|X|}
        \hline
        Syntax: & srav \$d, \$t, \$s \\
        \hline
        Description: & Shifts a register value right by the value in 
                        a second register and places the value in the 
                        destination register. The sign bit is shifted in. \\
        \hline
        Operation: & \$d = \$t >> \$s; \\
        \hline
        Encoding: & 0000 00ss ssst tttt dddd d000 0000 0111 \\
        \hline
    \end{tabularx}
    \end{table}

    \begin{table}[!hbp]
    \begin{tabularx}{\textwidth}{|c|X|}
        \hline
        Syntax: & srl \$d, \$t, h \\
        \hline
        Description: & Shifts a register value right by the shift
                         amount (shamt) and places the value in the 
                         destination register. Zeroes are shifted in. \\
        \hline
        Operation: & \$d = \$t >> h; \\
        \hline
        Encoding: & 0000 00-- ---t tttt dddd dhhh hh00 0010 \\
        \hline
    \end{tabularx}
    \end{table}

    \begin{table}[!hbp]
    \begin{tabularx}{\textwidth}{|c|X|}
        \hline
        Syntax: & srlv \$d, \$t, \$s \\
        \hline
        Description: & Shifts a register value right by the amount
                         specified in \$s and places the value in the 
                         destination register. Zeroes are shifted in. \\
        \hline
        Operation: & \$d = \$t >> \$s; \\
        \hline
        Encoding: & 0000 00ss ssst tttt dddd d000 0000 0110 \\
        \hline
    \end{tabularx}
    \end{table}

    \begin{table}[!hbp]
    \begin{tabularx}{\textwidth}{|c|X|}
        \hline
        Syntax: & beq \$s, \$t, offset \\
        \hline
        Description: & Branches if the two registers are equal. \\
        \hline
        Operation: & if \$s == \$t advance\_pc (offset << 2); else advance\_pc (4); \\
        \hline
        Encoding: & 0001 00ss ssst tttt iiii iiii iiii iiii \\
        \hline
    \end{tabularx}
    \end{table}

\clearpage  %the maximize number of picutres/tables without this command is 18
            %for more information, search for 'LaTeX Error: Too many unprocessed floats'

    \begin{table}[!hbp]
    \begin{tabularx}{\textwidth}{|c|X|}
        \hline
        Syntax: & bgez \$s, offset \\
        \hline
        Description: & Branches if the register is greater than or equal to zero. \\
        \hline
        Operation: & if \$s >= 0 advance\_pc (offset << 2)); else advance\_pc (4); \\
        \hline
        Encoding: & 0000 01ss sss0 0001 iiii iiii iiii iiii \\
        \hline
    \end{tabularx}
    \end{table}

    \begin{table}[!hbp]
    \begin{tabularx}{\textwidth}{|c|X|}
        \hline
        Syntax: & bgtz \$s, offset \\
        \hline
        Description: & Branches if the register is greater than zero. \\
        \hline
        Operation: & if \$s > 0 advance\_pc (offset << 2)); else advance\_pc (4); \\
        \hline
        Encoding: & 0001 11ss sss0 0000 iiii iiii iiii iiii \\
        \hline
    \end{tabularx}
    \end{table}

    \begin{table}[!hbp]
    \begin{tabularx}{\textwidth}{|c|X|}
        \hline
        Syntax: & blez \$s, offset \\
        \hline
        Description: & Branches if the register is less than or equal to zero. \\
        \hline
        Operation: & if \$s <= 0 advance\_pc (offset << 2)); else advance\_pc (4); \\
        \hline
        Encoding: & 0001 10ss sss0 0000 iiii iiii iiii iiii \\
        \hline
    \end{tabularx}
    \end{table}

    \begin{table}[!hbp]
    \begin{tabularx}{\textwidth}{|c|X|}
        \hline
        Syntax: & bltz \$s, offset \\
        \hline
        Description: & Branches if the register is less than zero. \\
        \hline
        Operation: & if \$s < 0 advance\_pc (offset << 2)); else advance\_pc (4); \\
        \hline
        Encoding: & 0000 01ss sss0 0000 iiii iiii iiii iiii \\
        \hline
    \end{tabularx}
    \end{table}

    \begin{table}[!hbp]
    \begin{tabularx}{\textwidth}{|c|X|}
        \hline
        Syntax: & bne \$s, \$t, offset \\
        \hline
        Description: & Branches if the two registers are not equal. \\
        \hline
        Operation: & if \$s != \$t advance\_pc (offset << 2)); else advance\_pc (4); \\
        \hline
        Encoding: & 0001 01ss ssst tttt iiii iiii iiii iiii \\
        \hline
    \end{tabularx}
    \end{table}

    \begin{table}[!hbp]
    \begin{tabularx}{\textwidth}{|c|X|}
        \hline
        Syntax: & j target \\
        \hline
        Description: & Jumps to the calculated address. \\
        \hline
        Operation: & PC = (PC \& 0xf0000000) | (target << 2) \\
        \hline
        Encoding: & 0000 10ii iiii iiii iiii iiii iiii iiii \\
        \hline
    \end{tabularx}
    \end{table}

    \begin{table}[!hbp]
    \begin{tabularx}{\textwidth}{|c|X|}
        \hline
        Syntax: & jal target \\
        \hline
        Description: & Jumps to the calculated address and stores the return address in \$RA \\
        \hline
        Operation: & \$RA = RPC; PC = (PC \& 0xf0000000) | (target << 2); \\
        \hline
        Encoding: & 0000 11ii iiii iiii iiii iiii iiii iiii \\
        \hline
    \end{tabularx}
    \end{table}

    \begin{table}[!hbp]
    \begin{tabularx}{\textwidth}{|c|X|}
        \hline
        Syntax: & jalr \$s, \$d \\
        \hline
        Description: & Jumps to the address of the first register,
                     and stores the return address in the second register \\
        \hline
        Operation: & \$d = RPC; PC = \$s \\
        \hline
        Encoding: & 0000 11ii iiii iiii iiii iiii iiii iiii \\
        \hline
    \end{tabularx}
    \end{table}

    \begin{table}[!hbp]
    \begin{tabularx}{\textwidth}{|c|X|}
        \hline
        Syntax: & jr \$s \\
        \hline
        Description: & Jump to the address contained in register \$s. \\
        \hline
        Operation: & PC = \$s; \\
        \hline
        Encoding: & 0000 00ss sss0 0000 0000 0000 0000 1000 \\
        \hline
    \end{tabularx}
    \end{table}

    \begin{table}[!hbp]
    \begin{tabularx}{\textwidth}{|c|X|}
        \hline
        Syntax: & lw \$t, offset(\$s) \\
        \hline
        Description: & A word is loaded into a register from the specified address. \\
        \hline
        Operation: & \$t = MEM[\$s + offset]; \\
        \hline
        Encoding: & 1000 11ss ssst tttt iiii iiii iiii iiii \\
        \hline
    \end{tabularx}
    \end{table}

    \begin{table}[!hbp]
    \begin{tabularx}{\textwidth}{|c|X|}
        \hline
        Syntax: & sw \$t, offset(\$s) \\
        \hline
        Description: & The contents of \$t is stored at the specified address. \\
        \hline
        Operation: & MEM[\$s + offset] = \$t; \\
        \hline
        Encoding: & 1010 11ss ssst tttt iiii iiii iiii iiii \\
        \hline
    \end{tabularx}
    \end{table}

    \begin{table}[!hbp]
    \begin{tabularx}{\textwidth}{|c|X|}
        \hline
        Syntax: & lb \$t, offset(\$s) \\
        \hline
        Description: & A byte is loaded into a register from the specified address. \\
        \hline
        Operation: & \$t = MEMByte[\$s + offset]; \\
        \hline
        Encoding: & 1000 00ss ssst tttt iiii iiii iiii iiii \\
        \hline
    \end{tabularx}
    \end{table}

    \begin{table}[!hbp]
    \begin{tabularx}{\textwidth}{|c|X|}
        \hline
        Syntax: & lbu \$t, offset(\$s) \\
        \hline
        Description: & A byte is loaded into a register from the specified address with zero extend. \\
        \hline
        Operation: & \$t = zero-extend( MEMByte[\$s + offset] ); \\
        \hline
        Encoding: & 1001 00ss ssst tttt iiii iiii iiii iiii \\
        \hline
    \end{tabularx}
    \end{table}

    \begin{table}[!hbp]
    \begin{tabularx}{\textwidth}{|c|X|}
        \hline
        Syntax: & sb \$t, offset(\$s) \\
        \hline
        Description: & The least significant byte of \$t is stored at the specified address. \\
        \hline
        Operation: & MEM[\$s + offset] = (0xff \& \$t); \\
        \hline
        Encoding: & 1010 00ss ssst tttt iiii iiii iiii iiii \\
        \hline
    \end{tabularx}
    \end{table}

    \begin{table}[!hbp]
    \begin{tabularx}{\textwidth}{|c|X|}
        \hline
        Syntax: & syscall \\
        \hline
        Description: & Generates a software interrupt. \\
        \hline
        Operation: & nothing \\
        \hline
        Encoding: & 0000 00-- ---- ---- ---- ---- --00 1100 \\
        \hline
    \end{tabularx}
    \end{table}

    \begin{table}[!hbp]
    \begin{tabularx}{\textwidth}{|c|X|}
        \hline
        Syntax: & cache \\
        \hline
        Description: & Nothing, alias to NOP \\
        \hline
        Operation: & nothing \\
        \hline
        Encoding: & 1011 11ss ssst tttt iiii iiii iiii iiii \\
        \hline
    \end{tabularx}
    \end{table}

\clearpage  %the maximize number of picutres/tables without this command is 18
            %for more information, search for 'LaTeX Error: Too many unprocessed floats'

    \begin{table}[!hbp]
    \begin{tabularx}{\textwidth}{|c|X|}
        \hline
        Syntax: & eret \\
        \hline
        Description: & Return from trap \\
        \hline
        Operation: & PC = EPC \\
        \hline
        Encoding: & 0100 0010 0000 0000 0000 0000 0001 1000 \\
        \hline
    \end{tabularx}
    \end{table}

    \begin{table}[!hbp]
    \begin{tabularx}{\textwidth}{|c|X|}
        \hline
        Syntax: & mfc0 \$t, \$d \\
        \hline
        Description: & Move the value of a CP0 register to a general register. \\
        \hline
        Operation: & \$t = CP0[\$d] \\
        \hline
        Encoding: & 0100 0000 000t tttt dddd d000 0000 0000 \\
        \hline
    \end{tabularx}
    \end{table}

    \begin{table}[!hbp]
    \begin{tabularx}{\textwidth}{|c|X|}
        \hline
        Syntax: & mtc0 \$d, \$t \\
        \hline
        Description: & Move the value of a general register to a CP0 register. \\
        \hline
        Operation: & CP0[\$d] = \$t \\
        \hline
        Encoding: & 0100 0000 100t tttt dddd d000 0000 0000 \\
        \hline
    \end{tabularx}
    \end{table}

    \begin{table}[!hbp]
    \begin{tabularx}{\textwidth}{|c|X|}
        \hline
        Syntax: & tlbwi \\
        \hline
        Description: & Write a TLB entry using Index, EntryHi, EntryLo0, EntryLo1 from CP0 register. \\
        \hline
        Operation: & Write a TLB entry. \\
        \hline
        Encoding: & 0100 0010 0000 0000 0000 0000 0000 0010 \\
        \hline
    \end{tabularx}
    \end{table}

    
    \begin{table}[!hbp]
    \begin{tabularx}{\textwidth}{|c|X|}
        \hline
        Syntax: & lhu \$t, \$s, imm \\
        \hline
        Description: & Load half word from memory to general register. \\
        \hline
        Operation: & \$t = zero-extend( MEMHalfWord[\$s + offset] ); \\
        \hline
        Encoding: & 1001 01ss ssst tttt iiii iiii iiii iiii \\
        \hline
    \end{tabularx}
    \end{table}

    