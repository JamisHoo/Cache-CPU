\section{运行环境需求}
    \subsection{设备}

        \begin{table}[!hbp]
        \centering
        \caption{设备与外部接口}
        \begin{tabular}{|l|l|}
        \hline
        环境 & 描述 \\
        \hline
        FPGA & Xilinx Spartan6 xc6slx100 \\
        \hline
        RAM & 32-bit字长, 4块, 共8MB \\
        \hline
        Flash & 16-bit 字长, 共8MB \\
        \hline
        CPLD & 与FPGA相连, 用于I/O \\
        \hline
        串口 & 2个 \\
        \hline
        ps/2 接口 & 1个 \\
        \hline
        以太网接口 & 1个 \\
        \hline
        VGA 接口 & 1个 \\
        \hline
        \end{tabular}
        \end{table}

    \subsection{控制}
        控制部分采用串口作为输入设备,%
        如果有需要可以启用实验板上的第二个串口。%
